\chapter{Συμπεράσματα και μελλοντική εργασία}
\label{chapter:conclusions}

Η παρούσα διπλωματική εργασία διερεύνησε τη βελτιστοποίηση συστημάτων Επαυξημένης Παραγωγής μέσω Ανάκτησης (Retrieval-Augmented Generation, RAG) στο πεδίο των τεχνικών ερωτημάτων μηχανικής λογισμικού, εστιάζοντας στη βελτίωση τόσο του υποσυστήματος ανάκτησης όσο και του υποσυστήματος παραγωγής. Μέσω τριών συστηματικών πειραμάτων και της πρότασης ενός μηχανισμού αυτοδιορθούμενης γενεσιουργίας, αναδείχθηκαν σημαντικά ευρήματα σχετικά με τις ικανότητες και τους περιορισμούς των σύγχρονων τεχνικών, καθώς και με τους θεμελιώδεις συμβιβασμούς που χαρακτηρίζουν τα συστήματα αυτά.

\section{Σύνθεση Πειραματικών Ευρημάτων}
\label{sec:synthesis_findings}

Η συγκριτική αξιολόγηση πέντε μεθόδων ανάκτησης (BM25, SPLADE, Dense BGE-M3, Hybrid SPLADE+BGE-M3, Hybrid BM25+BGE-M3) αποκάλυψε σαφή ιεραρχία απόδοσης που διαφοροποιείται ανάλογα με τη μετρική. Οι πυκνές μέθοδοι, και ιδιαίτερα η Dense BGE-M3, εμφάνισαν την υψηλότερη επίδοση στις μετρικές πρώιμης επιτυχίας και τις rank-aware μετρικές (MAP, MRR, NDCG@k), επιβεβαιώνοντας ότι η σημασιολογική ομοιότητα υπερέχει έναντι της λεξιλογικής αντιστοίχισης σε σύντομα τεχνικά τεκμήρια. Η SPLADE υπερείχε σαφώς της BM25, αποδεικνύοντας την αξία της μαθημένης επέκτασης όρων. Οι υβριδικές προσεγγίσεις δεν ξεπέρασαν την καθαρά πυκνή ανάκτηση με προεπιλεγμένες παραμέτρους σύμφυσης, γεγονός που υποδηλώνει την ανάγκη προσεκτικής παραμετροποίησης των βαρών των σημάτων. Παράλληλα, παρατηρήθηκε ότι οι λεξιλογικές μέθοδοι είναι ταχύτερες, ενώ οι πυκνές και υβριδικές εμφανίζουν μεγαλύτερη καθυστέρηση, χωρίς ωστόσο να αναιρούν τα οφέλη ποιότητας.

Το δεύτερο πείραμα επικεντρώθηκε στη βελτιστοποίηση των υπερπαραμέτρων του αλγορίθμου Reciprocal Rank Fusion (RRF). Η εξαντλητική αναζήτηση στον χώρο παραμέτρων $(\alpha, rrf_k)$ κατέδειξε ότι η βέλτιστη διαμόρφωση αντιστοιχεί σε υβριδική ανάκτηση με ισχυρή έμφαση στην πυκνή ενσωμάτωση ($\alpha^* = 0.8$, $rrf_k^* = 20$). Η παράμετρος $\alpha$ αποδείχθηκε κρίσιμη για την απόδοση, με τη σύνθετη βαθμολογία να αυξάνεται μονοτονικά έως το 0.8 και να σταθεροποιείται πέραν αυτού, ενώ το $rrf_k$ είχε αμελητέα επίδραση. Οι μετρικές Success@3 και Precision@3 αυξήθηκαν σημαντικά με το $\alpha$, ενώ η Recall@10 παρέμεινε σταθερή, επιβεβαιώνοντας ότι οι διαφορετικές μέθοδοι ανακτούν παρόμοιο πλήθος σχετικών εγγράφων αλλά διαφέρουν στην κατάταξη. Το αποτέλεσμα ερμηνεύεται από τα χαρακτηριστικά του συνόλου δεδομένων SOSum, όπου τα έγγραφα είναι σύντομα (διάμεσος 180 λεκτικές μονάδες), επιτρέποντας στα μοντέλα ενσωμάτωσης να αποτυπώνουν πλήρως το σημασιολογικό τους περιεχόμενο. Σε μακροσκελή έγγραφα, όπου απαιτείται τμηματοποίηση, η συνεισφορά του αραιού σήματος αναμένεται να είναι εντονότερη.

Το τρίτο πείραμα εστίασε στην αξιολόγηση του συστήματος από άκρη σε άκρη, συνδυάζοντας τη βελτιστοποιημένη ανάκτηση με τη διαδικασία παραγωγής απαντήσεων. Η αξιολόγηση πραγματοποιήθηκε μέσω της προσέγγισης LLM-as-a-Judge (αξιολογητής: GPT-5) σε 500 ερωτήματα, μετρώντας Πιστότητα, Συνάφεια και Χρησιμότητα. Το βασικό σύστημα RAG πέτυχε συνολική βαθμολογία 0.755 ± 0.177, με υψηλή Συνάφεια (0.873) αλλά χαμηλότερη Πιστότητα (0.677), γεγονός που αναδεικνύει το φαινόμενο των παραισθήσεων του μοντέλου Llama3.1 8B. Η εφαρμογή του προτεινόμενου μηχανισμού αυτοδιορθούμενης παραγωγής (Self-RAG) βελτίωσε την Πιστότητα κατά 2.5 ποσοστιαίες μονάδες (0.694), όμως μείωσε τη Χρησιμότητα κατά 6.6 ποσοστιαίες μονάδες (0.669), οδηγώντας σε ελαφρά μείωση της συνολικής απόδοσης. Το αποτέλεσμα αυτό αντικατοπτρίζει τον θεμελιώδη συμβιβασμό μεταξύ ακρίβειας και πληρότητας: η αυστηρή επαλήθευση μειώνει τις παραισθήσεις αλλά περιορίζει το πληροφοριακό εύρος των απαντήσεων. Επιπλέον, το μέγεθος του μοντέλου αποδείχθηκε καθοριστικός παράγοντας, καθώς τα μικρότερα μοντέλα εμφανίζουν περιορισμένες ικανότητες αυτοεπαλήθευσης και μετα-γνωστικής βελτίωσης.

Συνολικά, τα πειράματα ανέδειξαν τρία κρίσιμα συμπεράσματα:
\begin{itemize}
    \item Η πυκνή ανάκτηση αποτελεί το σημείο αναφοράς για σύντομα τεχνικά έγγραφα, με τα υβριδικά μοντέλα να προσφέρουν οριακές βελτιώσεις όταν ρυθμίζονται κατάλληλα.
    \item Η βελτιστοποίηση των υπερπαραμέτρων του RRF μπορεί να βελτιώσει σημαντικά την απόδοση, αλλά η σχετική επίδραση εξαρτάται από τα χαρακτηριστικά του πεδίου.
    \item Η απόδοση του υποσυστήματος παραγωγής απαντήσεων εξαρτάται όχι μόνο από την ποιότητα ανάκτησης αλλά και από το μέγεθος και τις γνωστικές ικανότητες του μοντέλου.
\end{itemize}
\section{Περιορισμοί της Παρούσας Έρευνας}
Η παρούσα έρευνα υπόκειται σε ορισμένους θεμελιώδεις περιορισμούς που επηρεάζουν την ερμηνεία και τη γενίκευση των αποτελεσμάτων, αντανακλώντας κοινές προκλήσεις στο πεδίο των συστημάτων επαυξημένης παραγωγής μέσω ανάκτησης (RAG).

\paragraph{Στοχαστικότητα και αναπαραγωγιμότητα των γλωσσικών μοντέλων.}
Η εγγενής στοχαστική φύση των μεγάλων γλωσσικών μοντέλων προκαλεί σημαντική μεταβλητότητα στις απαντήσεις, καθιστώντας δύσκολη την ακριβή αναπαραγωγή των πειραμάτων. Ακόμη και με σταθερές ρυθμίσεις, παρατηρούνται αποκλίσεις στις μετρικές απόδοσης μεταξύ επαναλήψεων, ενώ η χρήση LLMs ως κριτών (\textit{LLM-as-a-judge}) εισάγει επιπλέον αστάθεια και μεροληψία.

\paragraph{Περιορισμένη αξιολόγηση επανακατάταξης.}
Αν και το σύστημα υποστηρίζει μηχανισμούς επανακατάταξης μέσω διασταυρωμένων κωδικοποιητών (\textit{cross-encoders}), η συστηματική αξιολόγησή τους δεν πραγματοποιήθηκε λόγω υπολογιστικών περιορισμών. Προηγούμενες μελέτες \citep{nogueira2020monot5,fu2024autorag} έχουν δείξει ότι μοντέλα όπως τα \textit{MonoT5} και \textit{ColBERT} μπορούν να βελτιώσουν την ακρίβεια κατά 8--15\%, γεγονός που υποδεικνύει ότι τα ευρήματα της παρούσας μελέτης ενδέχεται να υποεκτιμούν το πραγματικό δυναμικό βελτίωσης.

\paragraph{Περιορισμοί του συνόλου δεδομένων.}
Η αξιολόγηση βασίστηκε σε ένα μόνο σύνολο δεδομένων (\textit{SOSum}) με σύντομα τεχνικά κείμενα, γεγονός που περιορίζει τη γενίκευση των συμπερασμάτων σε πεδία με μακροσκελή έγγραφα ή πολυεπίπεδη δομή πληροφορίας. Επιπλέον, η απουσία επισημασμένων σχέσεων συνάφειας (\textit{relevance labels}) εισάγει αβεβαιότητα σε μετρικές κατάταξης όπως το NDCG.

\paragraph{Συμβιβασμός μεταξύ πιστότητας και πληρότητας.}
Τα πειραματικά αποτελέσματα του μηχανισμού \textit{Self-RAG} ανέδειξαν την ύπαρξη εγγενούς συμβιβασμού μεταξύ ακρίβειας (\textit{faithfulness}) και πληροφοριακού πλούτου. Η αυξημένη πιστότητα συνοδεύεται από απώλεια χρησιμότητας, φαινόμενο που επιβεβαιώνει τον περιορισμό των τρεχουσών τεχνικών στην επίτευξη ισορροπίας μεταξύ αυστηρής επαλήθευσης και παραγωγικής ευελιξίας.

Οι παραπάνω περιορισμοί υπογραμμίζουν τις συστημικές προκλήσεις που αντιμετωπίζει το πεδίο και αναδεικνύουν την ανάγκη για μελλοντική έρευνα σε τομείς όπως η ποσοτικοποίηση της αβεβαιότητας, η ανάπτυξη σταθερότερων μηχανισμών παραγωγής και η δημιουργία συνόλων δεδομένων με πλήρεις επισημάνσεις συνάφειας.



\section{Μελλοντικές Κατευθύνσεις Έρευνας}
\label{sec:future_work}

Η εργασία ανοίγει πολλαπλές κατευθύνσεις για μελλοντική έρευνα και πρακτική ανάπτυξη:

\begin{enumerate}
    \item \textbf{Ενσωμάτωση του Πλαισίου AutoRAG.}  
    Η υιοθέτηση του πλαισίου AutoRAG θα επιτρέψει την αυτοματοποιημένη σύνθεση και βελτιστοποίηση αγωγών (retrievers, re-rankers, generators, evaluators), ενσωματώνοντας τεχνικές meta-optimization και multi-objective αξιολόγηση κόστους, ποιότητας και καθυστέρησης. Η προσέγγιση αυτή θα προσφέρει δυναμική επιλογή βέλτιστων διαμορφώσεων για διαφορετικά περιβάλλοντα χρήσης.

    \item \textbf{Online Βελτιστοποίηση με αλγόριθμο MAB.}
    Η ενσωμάτωση στρατηγικών τύπου \textit{Multi-Armed Bandit (MAB)} επιτρέπει την προσαρμοστική επιλογή υπερπαραμέτρων (π.χ. $\alpha$, $rrf_k$, top-k) και παραλλαγών prompt σε πραγματικό χρόνο, βελτιώνοντας σταδιακά την απόδοση του συστήματος με βάση ανατροφοδότηση από προηγούμενα ερωτήματα. Η διαδικασία στοχεύει στη μείωση της \textit{μεταμέλειας} (regret) υπό περιορισμούς καθυστέρησης (latency) και κόστους, σύμφωνα με τη λογική της online βελτιστοποίησης υπερπαραμέτρων.

    \item \textbf{Επέκταση σε Πολλαπλά Σύνολα Δεδομένων και Πεδία.}  
    Η εφαρμογή της προτεινόμενης μεθοδολογίας σε διαφορετικά πεδία, όπως επιστημονικές δημοσιεύσεις, τεχνική τεκμηρίωση και νομικά κείμενα, θα επιτρέψει τη γενίκευση των συμπερασμάτων και τη διερεύνηση της σχέσης μεταξύ μήκους εγγράφου και βέλτιστης ισορροπίας αραιών/πυκνών σημάτων.

    \item \textbf{Ανάπτυξη Γραφικού Περιβάλλοντος Πειραματισμού και Deployment Agent.}  
    Η δημιουργία γραφικού περιβάλλοντος (GUI) για τη διαχείριση αγωγών θα επιτρέψει τη διαδραστική σύνθεση, εκτέλεση και σύγκριση πειραμάτων με οπτικοποίηση μετρικών. Επιπλέον, η ενσωμάτωση ενός \textit{deployment agent} θα επιτρέπει την αυτόματη μεταφορά των πειραματικά βέλτιστων αγωγών σε παραγωγικό περιβάλλον με δυνατότητες παρακολούθησης, επαναφοράς σε προηγούμενη έκδοση (rollback) και A/B testing.

    \item \textbf{Βελτίωση του Μηχανισμού Self-RAG.}  
    Η περαιτέρω εξέλιξη του Self-RAG μπορεί να περιλαμβάνει διαβαθμισμένη επαλήθευση (διάκριση κρίσιμων και ήπιων αποκλίσεων), στρατηγικές \textit{edit-over-delete} κατά την αναθεώρηση, καθώς και πιο ευέλικτα κριτήρια τερματισμού του βρόχου επαλήθευσης. Η δοκιμή του μηχανισμού με ισχυρότερα μοντέλα (GPT-5-mini, Claude Haiku 4.5) και η αξιοποίηση μικρότερων εξειδικευμένων μοντέλων για την επαλήθευση αναμένεται να βελτιώσει την ισορροπία μεταξύ πιστότητας και χρησιμότητας.
\end{enumerate}

Οι παραπάνω κατευθύνσεις συνθέτουν ένα ολοκληρωμένο πλάνο εξέλιξης των συστημάτων RAG προς την κατεύθυνση της αυτοματοποίησης, της προσαρμοστικότητας και της πρακτικής αξιοποίησης, με στόχο τη δημιουργία αξιόπιστων και αναπαραγώγιμων υποδομών για την παραγωγή τεκμηριωμένης γνώσης.
