\begin{center}
  \centering

  \vspace{0.5cm}
  \centering
  \textbf{\Large{Περίληψη}}
  \phantomsection
  \addcontentsline{toc}{section}{Περίληψη}

  \vspace{1cm}

\end{center}

Η εκθετική αύξηση του όγκου των διαθέσιμων πληροφοριών στη σύγχρονη ψηφιακή εποχή, σε συνδυασμό με την ανάπτυξη προηγμένων μοντέλων τεχνητής νοημοσύνης, έχει δημιουργήσει νέες δυνατότητες αλλά και σημαντικές προκλήσεις στον τομέα της επεξεργασίας φυσικής γλώσσας. Τα μεγάλα γλωσσικά μοντέλα (LLMs), παρά τις εντυπωσιακές τους δυνατότητες στην παραγωγή φυσικού κειμένου, αντιμετωπίζουν κρίσιμους περιορισμούς που σχετίζονται με την ακρίβεια και την επικαιρότητα των πληροφοριών που παρέχουν. Το φαινόμενο των παραισθήσεων (hallucinations), όπου τα μοντέλα παράγουν φαινομενικά αξιόπιστες αλλά ανακριβείς πληροφορίες, καθώς και η αδυναμία πρόσβασης σε γνώση πέρα από τα δεδομένα εκπαίδευσής τους, περιορίζουν σημαντικά τη χρήση τους σε συστήματα που απαιτούν υψηλή αξιοπιστία.

Ανταποκρινόμενη σε αυτές τις προκλήσεις, η παρούσα εργασία εισάγει ένα επεκτάσιμο και ευπροσάρμοστο πλαίσιο πειραματισμού για συστήματα Επαυξημένης Παραγωγής μέσω Ανάκτησης (Retrieval-Augmented Generation, RAG), χρησιμοποιώντας ως πεδίο μελέτης τα τεχνικά ερωτήματα μηχανικής λογισμικού και αξιολογώντας πραγματικά δεδομένα. Το προτεινόμενο σύστημα σχεδιάστηκε με έμφαση στην επεκτασιμότητα, την αναπαραγωγισιμότητα και την πλήρη ιχνηλασιμότητα των πειραμάτων, επιτρέποντας τη συστηματική διερεύνηση διαφορετικών μεθόδων ανάκτησης και παραγωγής.

Πέρα από το ανωτέρω τεχνικό πλαίσιο, η εργασία προτείνει μια μεθοδολογία ιεραρχικής βελτιστοποίησης και αξιολόγησης από άκρη σε άκρη (end-to-end) που συνδυάζει μετρικές ανάκτησης με ποιοτική αποτίμηση των παραγόμενων απαντήσεων μέσω της προσέγγισης «μεγάλο γλωσσικό μοντέλο ως κριτής» (LLM-as-a-Judge). Εισάγεται επίσης αρχιτεκτονική αυτοδιορθούμενης παραγωγής μέσω ανάκτησης (Self-RAG), η οποία ενσωματώνει βρόγχο επαλήθευσης και αναθεώρησης για τη βελτίωση της πιστότητας των απαντήσεων. Η συγκεκριμένη προσέγγιση συμβάλλει στην κωδικοποίηση της μεθοδολογίας κατασκευής συστημάτων RAG που θα επιφέρουν τη βέλτιστη απόδοση σε παραγωγικό περιβάλλον.  

Συνολικά, η εργασία προτείνει στοχευμένες ενέργειες βελτιστοποίησης που συμβαδίζουν με την πρόσφατη βιβλιογραφία και ικανοποιούν τις σχεδιαστικές απαιτήσεις ενός παραγωγικού συστήματος. 

