{\fontfamily{cmr}\selectfont

\phantomsection
\addcontentsline{toc}{section}{Abstract}


\begin{center}
  \centering
  \textbf{\Large{Title}}
  \vspace{0.5cm}

  \textbf{\large{Architecture and Optimization of Retrieval Augmented Generation Systems in Technical Software Engineering Questions}}



  \vspace{1cm}

  \centering
  \textbf{Abstract}
\end{center}

The exponential growth in the volume of available information in the modern digital age, combined with the development of advanced artificial intelligence models, has created new opportunities but also significant challenges in the field of natural language processing. Large language models (LLMs), despite their impressive capabilities in generating natural text, face critical limitations related to the accuracy of the information they provide. The phenomenon of hallucinations , where models generate seemingly reliable but inaccurate information, as well as the inability to access knowledge beyond their training data, significantly limit their use in systems that require high reliability.

Responding to these challenges, this thesis introduces an extensible and adaptable experimentation framework for Retrieval-Augmented Generation (RAG) systems, using technical software engineering queries as a testing ground and evaluating real-world data. The proposed system was designed with an emphasis on extensibility, reproducibility, and full traceability of experiments, allowing for the systematic exploration of different retrieval and generation methods.

Beyond the above technical framework, the thesis proposes a hierarchical optimization and end-to-end evaluation methodology that combines retrieval metrics with qualitative assessment of the generated responses through the "large language model as a judge" (LLM-as-a-Judge) approach. It also introduces a self-correcting retrieval-based generation (Self-RAG) architecture, which incorporates a verification and review loop to improve response faithfulness. This approach contributes to the codification of the methodology for constructing RAG systems that will deliver optimal performance in a production environment.
 
Overall, the thesis proposes targeted optimization actions that are in line with recent literature and meet the design requirements of a production system.

\begin{flushright}
  \vspace{2cm}
  Spiros Chatzigeorgiou
  \\
  Electrical \& Computer Engineering Department,
  \\
  Aristotle University of Thessaloniki, Greece
  \\
  October 2025
\end{flushright}

}
