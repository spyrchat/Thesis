\section{Πείραμα 1: Επιλογή Μεθόδου Ανάκτησης}
\label{section:experiment1}
Το πρώτο πείραμα έχει ως απώτερο σκοπό την δημιουργία μίας αναφοράς για τις μεθόδους ανάκτησης μειώνοντας έτσι δραστικά των χώρο των επιθυμητών λύσεων.
\subsection{Μεθοδολογία Αξιολόγησης}
\label{subsection:evaluation_methodology}

\subsubsection*{Δημιουργία Συνόλου Αναφοράς σε Επίπεδο Τμημάτων}

Για την αξιολόγηση των συστημάτων ανάκτησης πληροφορίας, δημιουργήθηκε 
σύνολο αναφοράς (ground truth) σε επίπεδο τμημάτων εγγράφων (chunks), όχι σε 
επίπεδο ολόκληρων εγγράφων. Το σύνολο δεδομένων SOSum περιέχει 506 ερωτήματα, 
και κάθε έγγραφο (Stack Overflow answer) χωρίζεται σε τμήματα μεγέθους 500 
tokens με επικάλυψη 100 tokens. Για τη δημιουργία της χαρτογράφησης (mapping) από το πεδίο των εγγράφων στο πεδίο των τμημάτων, εφαρμόστηκε η ακόλουθη λογική: αν ένα 
έγγραφο $D$ χαρακτηρίζεται ως σχετικό για ένα ερώτημα $Q$ στο αρχικό 
dataset, τότε όλα τα τμήματα που προέρχονται από το $D$ θεωρούνται σχετικά 
για το $Q$.

\subsubsection*{Διαδικασία Αξιολόγησης}

Η αξιολόγηση πραγματοποιείται αποκλειστικά σε επίπεδο τμημάτων. Κάθε 
σύστημα ανάκτησης επιστρέφει τα top-$k$ τμήματα για κάθε ερώτημα. Ένα ανακτημένο τμήμα χαρακτηρίζεται ως σχετικό αν προέρχεται από έγγραφο που είναι σχετικό με το ερώτημα. Οι μετρικές απόδοσης (Precision@k, Recall@k, F1-score, MAP, MRR, NDCG@k, Success@k) υπολογίζονται 
με βάση τον αριθμό των σχετικών τμημάτων στα ανακτημένα αποτελέσματα. Το πείραμα γίνεται σε όλο το σύνολο των δεδομένων χωρίς κάποιον διαχωρισμό.

\subsection{Στρατηγικές Ανάκτησης}
\label{subsection:retrieval_strategies}

Αξιολογούνται πέντε βασικές στρατηγικές ανάκτησης χωρίς την εφαρμογή 
αναδιάταξης (reranking):

\begin{enumerate}
    \item \textbf{BM25 Baseline} - Παραδοσιακή λεξικογραφική μέθοδος 
    
    \item \textbf{SPLADE Baseline} - Νευρωνική αραιή μέθοδος με επέκταση 
    όρων (learned term expansion) 
    
    \item \textbf{Dense BGE-M3} - Πυκνή σημασιολογική ανάκτηση με 
    διανυσματικές ενσωματώσεις 
    
    \item \textbf{Hybrid SPLADE + BGE-M3 (Hybrid S)} - Υβριδική προσέγγιση με 
    συνδυασμό αραιών και πυκνών αναπαραστάσεων μέσω RRF 

    
    \item \textbf{Hybrid BM25 + BGE-M3 (Hybrid B)} - Παραδοσιακή υβριδική προσέγγιση 
    με συνδυασμό BM25 και πυκνής ανάκτησης μέσω RRF
\end{enumerate}

Για τις υβριδικές στρατηγικές, η σταθερά του RRF ορίζεται σε $k=60$ και οι 
δύο λίστες αποτελεσμάτων (αραιή και πυκνή) συνδυάζονται με ισοδύναμη 
συνεισφορά.

\subsection{Επιλογή Μετρικών}
\label{subsection:metrics_selection}

Η αξιολόγηση εστιάζει κυρίως σε μετρικές βασισμένες σε σύνολα (set-based 
metrics) όπως Precision@k (\ref{eq:precision}), Recall@k (\ref{eq:recall}), F1-score (\ref{eq:f1}) και MAP (\ref{eq:map}). Αυτή η επιλογή 
αιτιολογείται από τον περιορισμό του dataset: οι απαντήσεις του Stack 
Overflow ταξινομούνται με βάση ψηφοφορία της κοινότητας και όχι με χρήση αντικειμενικής αλήθειας (ground truth relevance) 
, εισάγοντας θόρυβο στην κατάταξη.

\subsection{Αποτελέσματα}

\begin{table}[H]
\centering
\small
\setlength{\tabcolsep}{6pt}
\renewcommand{\arraystretch}{1.15}
\begin{tabularx}{\textwidth}{@{}lRRRRRRR@{}}
\toprule
\textbf{Μέθοδος} &
\thead{Precision\\\normalfont @5} &
\thead{Recall\\\normalfont @5} &
\thead{F1\\\normalfont @5} &
\thread{MAP} &
\thread{MRR} &
\thead{NDCG\\\normalfont @5} &
\thead{Latency\\\normalfont (ms)} \\
\midrule
BM25     & 0.411 ± 0.326 & 0.205 ± 0.176 & 0.245 ± 0.179 & 0.226 ± 0.187 & 0.667 ± 0.411 & 0.458 ± 0.333 &  \textbf{19.6 ± 131.045} \\
SPLADE   & 0.585 ± 0.310 & 0.347 ± 0.232 & 0.390 ± 0.209 & 0.409 ± 0.256 & 0.846 ± 0.317 & 0.666 ± 0.301 & 118.3 ± 75.185 \\
Dense    & 0.649 ± 0.304 & \textbf{0.410 ± 0.249} & \textbf{0.449 ± 0.214} & \textbf{0.481 ± 0.249} & \textbf{0.927 ± 0.240} & \textbf{0.744 ± 0.266} & 564 ± 769.545
\\
Hybrid-Splade & \textbf{0.652 ± 0.297} & 0.396 ± 0.235 & 0.440 ± 0.201 & 0.468 ± 0.242 & 0.897 ± 0.262 & 0.736 ± 0.272 & 737.4 ± 905.823 \\
Hybrid-BΜ25 & 0.577 ± 0.296 & 0.349 ± 0.224 & 0.386 ± 0.186 & 0.398 ± 0.231 & 0.871 ± 0.276 & 0.657 ± 0.282 & 601.4 ± 786.131 \\
\bottomrule
\end{tabularx}
\caption{Αποτελέσματα αξιολόγησης retrievers: μέσος όρος ± τυπική απόκλιση για Precision@5, Recall@5, F1@5, MAP, MRR, NDCG@5 και Latency.}
\label{tab:retrieval_results}
\end{table}

\begin{figure}[H]
    \centering
    \includegraphics[width=1\linewidth]{images/chapter5/fig1_overall_performance.png}
    \caption{Συνολική συνοπτική απόδοση του συστήματος}
    \label{fig:experiment1-overall}
\end{figure}
Το διάγραμμα \ref{fig:experiment1-overall} παρουσιάζει τη συγκριτική απόδοση των πέντε στρατηγικών ανάκτησης 
σε τρεις βασικές μετρικές. Παρατηρούνται τα ακόλουθα:
Πρώτον, η στρατηγική πυκνής ανάκτησης (Dense BGE-M3) επιδεικνύει την 
υψηλότερη απόδοση σε όλες τις μετρικές (MAP=0.481, MRR=0.927, NDCG@10=0.703), 
υπερτερώντας ακόμα και των υβριδικών προσεγγίσεων. Αυτό υποδηλώνει ότι για το συγκεκριμένο σύνολο δεδομένων τεχνικών ερωτημάτων, η σημασιολογική 
ομοιότητα που αιχμαλωτίζουν οι πυκνές ενσωματώσεις είναι περισσότερο 
κρίσιμη από την ακριβή λεξιλογική αντιστοίχιση. Δεύτερον, η νευρωνική αραιή μέθοδος SPLADE υπερτερεί σημαντικά της 
παραδοσιακής BM25 (MAP: 0.409 έναντι 0.226), επιβεβαιώνοντας την αξία της 
μαθημένης επέκτασης όρων. Η BM25 παρουσιάζει την χειρότερη απόδοση σε όλες 
τις μετρικές, με απόκλιση που ξεπερνά το $50\%$ από την καλύτερη μέθοδο.
Τρίτον, οι υβριδικές στρατηγικές δεν επιτυγχάνουν να ξεπεράσουν την πυκνή ανάκτηση, με τη διαμόρφωση Hybrid-S (SPLADE+Dense) να πλησιάζει αλλά να 
υστερεί ελαφρώς (MAP: 0.468 έναντι 0.481). Αυτό μπορεί να οφείλεται στη 
χρήση ίσων βαρών (0.5/0.5) στον αλγόριθμο RRF, που ενδεχομένως δεν είναι βέλτιστη για το συγκεκριμένο τομέα, ή στο ότι η προσθήκη αραιών ενσωματώσεων 
εισάγει θόρυβο χωρίς να προσθέτει συμπληρωματική πληροφορία.



\begin{figure}[H]
    \centering
    \includegraphics[width=1\linewidth]{images/chapter5/fig2_precision_at_k.png}
    \caption{Σύγκριση των διαμορφώσεων ως προς την μετρική: ακρίβεια}
    \label{fig:experiment1-precision}
\end{figure}

\begin{figure}[H]
    \centering
    \includegraphics[width=1\linewidth]{images/chapter5/fig3_recall_at_k.png}
    \caption{Σύγκριση των διαμορφώσεων ως προς την μετρική: ανάκληση}
    \label{fig:experiment1-recall}
\end{figure}

\begin{figure}[H]
    \centering
    \includegraphics[width=1\linewidth]{images/chapter5/fig4_f1_scores.png}
    \caption{Σύγκριση των διαμορφώσεων ως προς την μετρική: F1}
    \label{fig:experiment1-f1}
\end{figure}

\begin{figure}[H]
    \centering
    \includegraphics[width=0.75\linewidth]{images/chapter5/fig5_precision_recall_tradeoff.png}
    \caption{Συμβιβασμός μεταξύ ανάκλησης και ακρίβειας}
    \label{fig:experiment1-tradeoff}
\end{figure}

\begin{figure}[H]
    \centering
    \includegraphics[width=1\linewidth]{images/chapter5/fig6_ndcg_progression.png}
    \caption{Σύγκριση των διαμορφώσεων ως προς την μετρική: NDCG}
    \label{fig:experiment1-ndcg}
\end{figure}

Από το Σχήμα \ref{fig:experiment1-precision} (Precision@k) παρατηρείται ότι η πυκνή ανάκτηση (Dense) 
διατηρεί σταθερά την υψηλότερη ακρίβεια για όλες τις τιμές k, με ιδιαίτερα 
υψηλή απόδοση για k=1 που υποδηλώνει εξαιρετική ικανότητα τοποθέτησης 
σχετικών τμημάτων στην πρώτη θέση. Η αναμενόμενη πτώση της ακρίβειας καθώς 
αυξάνει το k είναι συνεπής με τη θεωρία, καθώς η συμπερίληψη περισσότερων 
αποτελεσμάτων αυξάνει την πιθανότητα εισαγωγής μη σχετικών τμημάτων.

Αντίστροφα, το Σχήμα \ref{fig:experiment1-recall} (Recall@k) επιβεβαιώνει τη θεωρητικά αναμενόμενη 
αύξηση της ανάκλησης με το k. Σημαντικό είναι ότι η χαμηλή ανάκληση για 
k=1 δεν αποτελεί αδυναμία του συστήματος, αλλά αντανακλά το γεγονός ότι 
υπάρχουν πολλαπλά σχετικά τμήματα ανά ερώτημα.

Το Σχήμα \ref{fig:experiment1-tradeoff} παρέχει μια συνοπτική απεικόνιση της σχέσης Precision-Recall 
για k=5, με τη Dense και Hybrid-S να βρίσκονται στην επιθυμητή περιοχή 
υψηλής απόδοσης. Η οριζόντια διασπορά των σημείων υποδηλώνει ότι οι 
στρατηγικές διαφοροποιούνται περισσότερο στην ικανότητά τους να ανακτούν 
ποσοστό των διαθέσιμων σχετικών τμημάτων παρά στην ακρίβεια των επιλογών 
τους.

Τέλος, το Σχήμα \ref{fig:experiment1-ndcg} (NDCG@k) επιβεβαιώνει τη συνολική υπεροχή της Dense 
μεθόδου σε rank-aware μετρικές. Η απότομη πτώση όλων των μεθόδων από k=1 
σε k=3 υποδηλώνει ότι η ποιότητα των αποτελεσμάτων υποβαθμίζεται γρήγορα 
πέρα από τις πρώτες θέσεις, ενώ η σχετικά επίπεδη καμπύλη από k=5 έως k=10 
υποδηλώνει σταθεροποίηση της κατάταξης.
\begin{figure}[H]
    \centering
    \includegraphics[width=1\linewidth]{images/chapter5/fig7_latency_analysis.png}
    \caption{Σύγκριση των διαμορφώσεων ως προς την μετρική: καθυστέρηση}
    \label{fig:experiment1-latency}
\end{figure}

Το Σχήμα \ref{fig:experiment1-latency} παρουσιάζει τη σύγκριση των πέντε στρατηγικών ανάκτησης ως προς 
την υπολογιστική απόδοση, αποκαλύπτοντας σημαντικές διαφορές στην καθυστέρηση 
εκτέλεσης.

Από το αριστερό διάγραμμα (Μέση Καθυστέρηση) παρατηρείται μια σαφής ιεραρχία 
απόδοσης. Η BM25 επιδεικνύει εξαιρετική ταχύτητα (19.6 ms), αναμενόμενο 
αποτέλεσμα δεδομένου ότι βασίζεται σε απλούς υπολογισμούς στατιστικής 
συχνότητας χωρίς νευρωνικά μοντέλα. Το SPLADE, παρότι αποτελεί νευρωνική 
προσέγγιση, διατηρεί ανταγωνιστική ταχύτητα (118.3 ms) λόγω της αραιότητας 
των διανυσμάτων του.

Η πυκνή ανάκτηση (Dense) παρουσιάζει σημαντικά υψηλότερη καθυστέρηση (623.3 ms), 
που οφείλεται στον υπολογισμό πυκνών διανυσμάτων 1024 διαστάσεων και στην 
εκτέλεση αναζήτησης πλησιέστερων γειτόνων σε μεγάλο χώρο. Οι υβριδικές 
στρατηγικές εμφανίζουν ακόμα υψηλότερη καθυστέρηση (Hybrid-S: 737.4 ms, 
Hybrid-B: 601.4 ms), αναμενόμενο αφού εκτελούν τόσο αραιή όσο και πυκνή 
αναζήτηση παράλληλα και στη συνέχεια εφαρμόζουν τον αλγόριθμο RRF για 
συνδυασμό των αποτελεσμάτων.

Το δεξί διάγραμμα (Εκατοστημόρια Καθυστέρησης) αποκαλύπτει επιπλέον 
σημαντικές πληροφορίες για την ευστάθεια της απόδοσης. Η διαφορά μεταξύ 
διάμεσου (P50) και P95 για τις μεθόδους με νευρωνικά μοντέλα είναι 
εντυπωσιακά μεγάλη, ειδικά για τις υβριδικές προσεγγίσεις όπου το P95 
φτάνει περίπου τα 1600 ms. Αυτή η υψηλή διασπορά υποδηλώνει ότι η 
καθυστέρηση δεν είναι σταθερή και επηρεάζεται από παράγοντες όπως η 
πολυπλοκότητα του ερωτήματος, το μέγεθος του ευρετηρίου που εξερευνάται, καθώς και την κατανάλωση των επεξεργαστικών πόρων τη δεδομένη χρονική στιγμή. Σημαντικό είναι ότι το διάγραμμα υποδηλώνει σαφή συμβιβασμό (tradeoff) μεταξύ απόδοσης 
ανάκτησης και υπολογιστικής αποδοτικότητας. Η Dense, που επιδεικνύει την καλύτερη απόδοση σε μετρικές ποιότητας (MAP, MRR, NDCG), απαιτεί περίπου 35 φορές περισσότερο χρόνο από την BM25, ενώ οι υβριδικές στρατηγικές, που 
προσεγγίζουν την Dense σε ποιότητα, απαιτούν ακόμα μεγαλύτερο υπολογιστικό 
κόστος. Για παραγωγικές εφαρμογές, αυτή η ανάλυση υποδηλώνει ότι η επιλογή 
στρατηγικής εξαρτάται από τις απαιτήσεις του συστήματος: για εφαρμογές 
real-time με αυστηρά όρια καθυστέρησης, η BM25 ή το SPLADE μπορεί να είναι 
προτιμότερα παρά την κάποια θυσία ποιότητας, ενώ για εφαρμογές όπου η 
ακρίβεια υπερτερεί της ταχύτητας, η Dense παραμένει η βέλτιστη επιλογή.