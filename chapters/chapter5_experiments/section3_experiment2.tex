\section{Πείραμα 2: Βελτιστοποίηση Υβριδικής Ανάκτησης}
\label{section:experiment2}

Τα αποτελέσματα του Πειράματος 1 κατέδειξαν ότι οι υβριδικές στρατηγικές, 
παρά τη θεωρητική τους υπεροχή λόγω του συνδυασμού συμπληρωματικών σημάτων, 
δεν κατάφεραν να υπερβούν την πυκνή ανάκτηση με τις προεπιλεγμένες 
παραμέτρους. Αυτή η παρατήρηση εγείρει το ερώτημα εάν η υποδεέστερη απόδοση 
οφείλεται σε μη βέλτιστη παραμετροποίηση του αλγορίθμου Reciprocal Rank 
Fusion. Το παρόν πείραμα διερευνά συστηματικά την επίδραση δύο κρίσιμων 
υπερπαραμέτρων στην απόδοση των υβριδικών μεθόδων: του βάρους σύμφυσης 
$\alpha$ και της σταθεράς RRF $rrf\_k$. Η μαθηματική περιγραφή του αλγορίθμου (\ref{alg:rrf}) που χρησιμοποιήθηκε είναι:

\begin{equation}
    \text{RRF}_{\text{hybrid}}(d) = \alpha \cdot \frac{1}{k + \text{rank}_{\text{dense}}(d)} + (1-\alpha) \cdot \frac{1}{k + \text{rank}_{\text{sparse}}(d)}
\end{equation}
\subsection{Μεθοδολογία Βελτιστοποίησης}
\label{subsection:optimization_methodology}

Η βελτιστοποίηση διατυπώνεται ως πρόβλημα εύρεσης του βέλτιστου ζεύγους 
υπερπαραμέτρων που μεγιστοποιεί μια σύνθετη συνάρτηση στόχου (composite 
objective function). Η παράμετρος $\alpha$ ελέγχει την ισορροπία μεταξύ 
πυκνών και αραιών αναπαραστάσεων, όπου $\alpha \in [0, 1]$ με $\alpha = 0$ 
να αντιστοιχεί σε καθαρά αραιή ανάκτηση, $\alpha = 1$ σε καθαρά πυκνή, και 
ενδιάμεσες τιμές σε υβριδική προσέγγιση. Η παράμετρος $rrf\_k$ αποτελεί τη 
σταθερά κανονικοποίησης του αλγορίθμου Reciprocal Rank Fusion που ελέγχει 
τη βαρύτητα των κορυφαίων αποτελεσμάτων. Το πρόβλημα βελτιστοποίησης 
διατυπώνεται τυπικά ως:

\begin{equation}
(\alpha^*, rrf\_k^*) = argmax_{(\alpha, rrf\_k) \in \Theta} S(\alpha, rrf\_k; D_{train})
\end{equation}

όπου $S$ είναι η συνάρτηση σύνθετης βαθμολογίας που υπολογίζεται στο σύνολο εκπαίδευσης $D_{train}$, και $\Theta$ ο χώρος αναζήτησης των υπερπαραμέτρων.Ο χώρος αναζήτησης ορίζεται ως το καρτεσιανό 
γινόμενο $\Theta = \{(\alpha_i, k_j) : \alpha_i \in A, k_j \in K\}$, όπου 
$A = \{0.0, 0.2, 0.4, 0.6, 0.8, 1.0\}$ αποτελεί ομοιόμορφη διακριτοποίηση 
του διαστήματος $[0, 1]$ με βήμα 0.2, και $K = \{20, 40, 60, 80, 100\}$ 
περιλαμβάνει κοινές τιμές από τη βιβλιογραφία. Αυτός ο ορισμός παράγει 
$|\Theta| = 30$ συνολικές διαμορφώσεις, επιτρέποντας την εξαντλητική 
αναζήτηση (exhaustive grid search) χωρίς υπερβολικό υπολογιστικό κόστος. Η 
επιλογή εξαντλητικής αναζήτησης αντί μεθόδων βασισμένων σε κλίση (gradient-based methods) αιτιολογείται 
από την πιθανή μη-κυρτότητα της συνάρτησης στόχου και την ανάγκη πλήρους 
χαρτογράφησης του χώρου για ερμηνευσιμότητα.
Η συνάρτηση σύνθετης βαθμολογίας (composite score) σχεδιάστηκε για να 
εξισορροπεί πολλαπλά κριτήρια απόδοσης, συνδυάζοντας βαθμολογία ποιότητας 
(quality score) με ποινή καθυστέρησης (latency penalty). Συγκεκριμένα, 
ορίζεται ως:

\begin{equation}
S(\alpha, rrf\_k) = Q(\alpha, rrf\_k) - P(t)
\end{equation} 

όπου η βαθμολογία ποιότητας $Q$ υπολογίζεται ως σταθμισμένος συνδυασμός 
τεσσάρων μετρικών:

\begin{equation}
Q = 0.35 \cdot \text{Success@3} + 0.30 \cdot \text{Precision@3} + 
    0.20 \cdot \text{Recall@10} + 0.15 \cdot \text{Precision@10}
\end{equation}

Οι σταθμίσεις επιλέχθηκαν έτσι ώστε να δίνεται μεγαλύτερη έμφαση στην 
πρώιμη επιτυχία (early success) και την ακρίβεια στα πρώτα αποτελέσματα, 
καθώς οι χρήστες συστημάτων RAG εξετάζουν κυρίως τις πρώτες θέσεις. Η ποινή 
καθυστέρησης ορίζεται ως:

\begin{equation}
P(t) = 0.1 \cdot \frac{\min(\max(0, t - t_{target}), t_{max})}{t_{max}}
\end{equation}

όπου $t$ είναι ο πραγματικός χρόνος απόκρισης σε milliseconds, $t_{target} = 500$ ms ο στόχος καθυστέρησης, και $t_{max} = 1000$ ms το ανώτατο όριο ποινής. Η ποινή είναι μηδενική για 
$t \leq 500$ ms, αυξάνεται γραμμικά έως 0.1 για χρόνους μέχρι 1500 ms, 
και παραμένει σταθερή στο 0.1 για μεγαλύτερες καθυστερήσεις, διασφαλίζοντας 
ότι η ποιότητα παραμένει το κυρίαρχο κριτήριο ενώ λαμβάνεται υπόψη η 
πρακτική εφαρμογή.

\textbf{Στρατηγική Διαχωρισμού Δεδομένων.} Για την αξιόπιστη εκτίμηση 
της απόδοσης και την επιλογή υπερπαραμέτρων που γενικεύουν σε νέα δεδομένα, 
το σύνολο δεδομένων SOSum χωρίζεται σε σύνολο εκπαίδευσης και σύνολο αξιολόγησης με αναλογία 80-20 μέσω στρωματοποιημένης τυχαίας δειγματοληψίας (stratified random split). Η επιλογή αυτής της μεθοδολογίας 
αιτιολογείται από το γεγονός ότι το σύνολο δεδομένων SOSum αποτελεί ένα 
αντιπροσωπευτικό υποσύνολο του ευρύτερου προβλήματος των τεχνικών 
ερωτημάτων στον τομέα της τεχνολογίας λογισμικού. Στόχος του διαχωρισμού είναι η εύρεση διαμόρφωσης υπερπαραμέτρων που επιδεικνύει ισχυρή 
ικανότητα γενίκευσης στο ευρύτερο πρόβλημα, αντί της απλής προσαρμογής στα 
συγκεκριμένα χαρακτηριστικά του διαθέσιμου συνόλου εκπαίδευσης.

Η στρωματοποίηση (stratification) διασφαλίζει ότι τόσο το σύνολο εκπαίδευσης όσο και το σύνολο αξιολόγησης διατηρούν 
την αναλογική κατανομή των ερωτημάτων σε σχέση με τρεις κρίσιμες διαστάσεις. 
Πρώτον, ως προς τον τύπο ερωτήματος (question type) που καθορίζει τη φύση 
της πληροφορίας που αναζητείται. Δεύτερον, ως προς την κύρια κατηγορία 
τεχνολογίας (primary tag category) που αντλείται από τις ετικέτες του Stack 
Overflow και ομαδοποιείται σε έξι συχνότερες κατηγορίες συν μία κατηγορία 
"Other" για τις υπόλοιπες. Τρίτον, ως προς το πλήθος διαθέσιμων απαντήσεων 
ανά ερώτημα, που χωρίζεται σε τέσσερα διακριτά επίπεδα (bins): "none" για 
ερωτήματα χωρίς απαντήσεις, "low" για 1-3 απαντήσεις, "medium" για 4-6 
απαντήσεις, και "high" για 7 ή περισσότερες απαντήσεις. Το πλήθος 
απαντήσεων αποτελεί σημαντική διάσταση στρωματοποίησης καθώς σχετίζεται με 
την πολυπλοκότητα του ερωτήματος και τη διαθεσιμότητα σχετικού περιεχομένου 
στη βάση γνώσης.

Το κλειδί στρωματοποίησης (stratification key) κατασκευάζεται ως η 
συνένωση των τριών αυτών διαστάσεων, δημιουργώντας διακριτές ομάδες με 
συγκεκριμένα χαρακτηριστικά. Για την αποφυγή υπερβολικά μικρών ομάδων που 
δεν μπορούν να χωριστούν ισορροπημένα, διατηρούνται μόνο οι ομάδες 
με ελάχιστο πλήθος τουλάχιστον 10 ερωτημάτων, επιτρέποντας την ύπαρξη 
τουλάχιστον 8 ερωτημάτων στο σύνολο εκπαίδευσης και 2 στο σύνολο αξιολόγησης. Αυτή η φιλτραριστική διαδικασία 
διατηρεί την πλειονότητα των δεδομένων ενώ εξασφαλίζει στατιστικά 
αξιόπιστες εκτιμήσεις.

\textbf{Διαδικασία Grid Search.} Για κάθε διαμόρφωση $(\alpha_i, k_j)$ του χώρου αναζήτησης $\Theta$, υπολογίζεται 
η βαθμολογία $S(\alpha_i, k_j; D_{train})$ στο σύνολο εκπαίδευσης. Η βέλτιστη διαμόρφωση επιλέγεται με ιεραρχική στρατηγική 
που μεγιστοποιεί πρωτίστως την τιμή της συνάρτησης στόχου, με δευτερεύον 
κριτήριο την προτίμηση ισορροπημένων τιμών $\alpha$ κοντά στο 0.5 (καθώς 
υβριδικές μέθοδοι που αξιοποιούν και τις δύο ενσωματώσεις τείνουν να γενικεύουν 
καλύτερα) και τριτεύον κριτήριο την προτίμηση της standard τιμής 
$rrf\_k = 60$ που αποτελεί κοινή πρακτική στη βιβλιογραφία.
Τέλος, η επιλεγμένη διαμόρφωση $(\alpha^*, rrf\_k^*)$ αξιολογείται στο σύνολο αξιολόγησης $D_{test}$ για αμερόληπτη εκτίμηση της απόδοσης σε δεδομένα που δεν έχουν 
χρησιμοποιηθεί κατά τη διαδικασία βελτιστοποίησης, παρέχοντας ρεαλιστική 
εκτίμηση της αναμενόμενης συμπεριφοράς σε νέα ερωτήματα. Η τελική αναφερόμενη απόδοση υπολογίζεται αποκλειστικά στο $D_{test}$ και αποτελεί την εκτίμηση της ικανότητας γενίκευσης του συστήματος.
\newpage
\subsection{Αποτελέσματα}
\label{subsection:experiment2_results}

\begin{figure}[H]
    \centering
    \includegraphics[width=1\linewidth]{images/chapter5/heatmap_composite_score.png}
    \caption{Χάρτης σύνθετης βαθμολογίας: η βέλτιστη διαμόρφωση ($\alpha=0.8$, $k=20$) αντιστοιχεί σε υβριδική ανάκτηση με ισχυρή έμφαση στο πυκνό σήμα. Η απόδοση αυξάνεται μονοτονικά με το $\alpha$, ενώ το $k$ έχει αμελητέα επίδραση.}
    \label{fig:heatmap-composite}
\end{figure}

Το Σχήμα \ref{fig:heatmap-composite} παρουσιάζει τον χάρτη σύνθετης 
βαθμολογίας στον δισδιάστατο χώρο αναζήτησης των υπερπαραμέτρων $\alpha$ 
και $k$. Η βέλτιστη διαμόρφωση εντοπίζεται στο $\alpha^* = 0.8$ και 
$k^* = 20$ με σύνθετη βαθμολογία $S = 0.6339$, υποδεικνύοντας ότι η 
υβριδική στρατηγική με ισχυρή έμφαση στην πυκνή ανάκτηση (80\% dense, 20\% sparse) επιτυγχάνει την υψηλότερη απόδοση. Παρατηρείται σαφής διαβάθμιση από χαμηλές βαθμολογίες 
(σκούρο κόκκινο, ~0.54-0.55) για $\alpha \leq 0.2$ προς υψηλές βαθμολογίες 
(πράσινο, ~0.63-0.64) για $\alpha \geq 0.6$, με το κρίσιμο σημείο μετάβασης 
να βρίσκεται περίπου στο $\alpha = 0.4$. Η παράμετρος $k$ επιδεικνύει 
αμελητέα επίδραση στην απόδοση, με όλες οι τιμές στο εύρος 20-100 να 
παράγουν σχεδόν ταυτόσημες βαθμολογίες για το ίδιο $\alpha$. Αξιοσημείωτο είναι ότι η περαιτέρω αύξηση του $\alpha$ πέρα από το 0.8 (π.χ. στο $\alpha=1.0$ που αντιστοιχεί σε καθαρά πυκνή ανάκτηση) δεν φαίνεται να προσφέρει επιπλέον βελτίωση, υποδηλώνοντας ότι η μικρή συνεισφορά του αραιού σήματος (20\%) παίζει κάποιο ρόλο στη βέλτιστη απόδοση.

\begin{figure}[H]
    \centering
    \includegraphics[width=1\linewidth]{images/chapter5/heatmaps_individual_metrics.png}
    \caption{Επιμέρους μετρικές: Success@3 και Precision@3 βελτιώνονται 
δραματικά με το $\alpha$, η Recall@10 παραμένει σταθερή (~0.51), και η 
καθυστέρηση εμφανίζει μικρές διακυμάνσεις. Το $k$ δεν επηρεάζει 
συστηματικά τις μετρικές.}
    \label{fig:heatmap-individual}
\end{figure}

Το Σχήμα \ref{fig:heatmap-individual} αποσυνθέτει τη σύνθετη 
βαθμολογία στις επιμέρους μετρικές που τη συνθέτουν, αποκαλύπτοντας τους 
υποκείμενους μηχανισμούς της παρατηρούμενης απόδοσης. Η μετρική Success@3 
εμφανίζει δραματική βελτίωση από 0.775 για $\alpha = 0$ σε 0.875 για 
$\alpha = 0.8$, επιβεβαιώνοντας την υπεροχή της πυκνής ανάκτησης στον 
εντοπισμό σχετικών τμημάτων στις πρώτες θέσεις. Η Precision@3 ακολουθεί 
παρόμοια τάση, αυξανόμενη από 0.535 σε 0.629, ενώ η Recall@10 παραμένει 
αξιοσημείωτα σταθερή στο ~0.51 ανεξάρτητα από το $\alpha$, υποδηλώνοντας 
ότι όλες οι στρατηγικές ανακτούν παρόμοιο ποσοστό των διαθέσιμων σχετικών 
εγγράφων όταν εξετάζονται τα πρώτα δέκα αποτελέσματα. Η Precision@10 
σταθεροποιείται στο ~0.361 για $\alpha \geq 0.6$, αλλά υποβαθμίζεται 
σημαντικά για χαμηλότερες τιμές (~0.318). Η βαθμολογία ποιότητας αντανακλά τη 
σταθμισμένη συνεισφορά αυτών των μετρικών, με την κυριαρχία των early 
success μετρικών να οδηγεί στην προτίμηση υψηλού $\alpha$. Τέλος, η 
καθυστέρηση εμφανίζει περιορισμένη διακύμανση, με αραιές μεθόδους 
($\alpha = 0$) να επιτυγχάνουν χρόνους απόκρισης ~906 ms έναντι υβριδικών και πυκνών μεθόδων (~830-840 ms για $\alpha \geq 0.6$). Η απόλυτη διαφορά (~70-80 ms) παραμένει σχετικά μικρή σε σχέση με τις βελτιώσεις 
ποιότητας. Η παράμετρος $k$ δεν επιδεικνύει συστηματική επίδραση σε 
καμία από τις επιμέρους μετρικές, επιβεβαιώνοντας ότι ο κύριος παράγοντας 
απόδοσης είναι η ισορροπία πυκνών-αραιών σημάτων μέσω του $\alpha$.

\begin{figure}[H]
    \centering
    \includegraphics[width=1\linewidth]{images/chapter5/surface_plot_3d.png}
    \caption{Τρισδιάστατη επιφάνεια σύνθετης βαθμολογίας στον χώρο 
υπερπαραμέτρων. Η επιφάνεια εμφανίζει μονοτονική αύξηση με το $\alpha$ και 
επιπεδότητα κατά μήκος του $k$. Το βέλτιστο σημείο ($\alpha=0.8$, $k=20$) επισημαίνεται με αστέρι.}
    \label{fig:experiment2-3d-surface}
\end{figure}

Το Σχήμα \ref{fig:experiment2-3d-surface} απεικονίζει την τρισδιάστατη 
επιφάνεια της σύνθετης βαθμολογίας στον χώρο των υπερπαραμέτρων, 
προσφέροντας γεωμετρική διαίσθηση της τοπολογίας της συνάρτησης στόχου. Η 
επιφάνεια εμφανίζει σαφή μονοτονική αύξηση κατά μήκος του άξονα $\alpha$, 
με το πράσινο οροπέδιο στην περιοχή $\alpha \geq 0.6$ να αντιστοιχεί στη 
ζώνη υψηλής απόδοσης. Αντίθετα, κατά μήκος του άξονα $k$, η 
επιφάνεια παραμένει σχεδόν επίπεδη, επιβεβαιώνοντας την αμελητέα επίδραση 
αυτής της παραμέτρου. Η βέλτιστη διαμόρφωση, επισημασμένη με αστέρι στο 
σημείο ($\alpha=0.8$, $k=20$), βρίσκεται στο πράσινο οροπέδιο με βαθμολογία ~0.63. Η οπτικοποίηση αποκαλύπτει την απουσία 
τοπικών μεγίστων στο εσωτερικό του χώρου αναζήτησης, με την απόδοση να αυξάνεται σταθερά καθώς το $\alpha$ προσεγγίζει το 0.8, υποδηλώνοντας ότι η σημαντική έμφαση στην πυκνή ενσωμάτωση με μικρή συμπληρωματική συνεισφορά από το αραιό σήμα αποδίδει τα καλύτερα αποτελέσματα.

Το Σχήμα \ref{fig:experiment2-sensitivity} παρουσιάζει την ανάλυση 
ευαισθησίας (sensitivity analysis) των υπερπαραμέτρων, διαχωρίζοντας την 
επίδραση κάθε παραμέτρου ξεχωριστά. Το αριστερό διάγραμμα απεικονίζει την 
ευαισθησία ως προς το $\alpha$ για διαφορετικές σταθερές τιμές $k$. 
Παρατηρείται ότι όλες οι καμπύλες ακολουθούν σχεδόν ταυτόσημη πορεία, με 
απότομη αύξηση της βαθμολογίας από ~0.54 στο $\alpha=0$ έως ~0.63 στο 
$\alpha=0.8$, και με το κρίσιμο σημείο καμπής να εντοπίζεται περίπου στο 
$\alpha=0.4$. Η διακεκομμένη κόκκινη γραμμή στο $\alpha=0.8$ επισημαίνει το βέλτιστο σημείο. Η σύγκλιση των καμπυλών επιβεβαιώνει ότι η επιλογή του 
$k$ δεν επηρεάζει ουσιαστικά τη σχέση απόδοσης-$\alpha$. Αξιοσημείωτο είναι ότι η απόδοση φαίνεται να σταθεροποιείται ή ακόμα και να υποχωρεί ελαφρώς για $\alpha > 0.8$, υποδηλώνοντας ότι η μικρή συνεισφορά της αραιής ενσωμάτωσης (20\%) είναι ωφέλιμη.

Το δεξί διάγραμμα εξετάζει την ευαισθησία ως προς το $k$ για 
διαφορετικές σταθερές τιμές $\alpha$. Οι καμπύλες εμφανίζονται σχεδόν 
οριζόντιες για όλες τις τιμές $\alpha$, με την απόδοση να παραμένει 
ουσιαστικά σταθερή σε όλο το εύρος $k \in [20, 100]$. Η 
διαστρωμάτωση των καμπυλών ανάλογα με το $\alpha$ επιβεβαιώνει την 
κυριαρχία αυτής της παραμέτρου: οι καμπύλες για $\alpha \geq 0.8$ 
(κόκκινες αποχρώσεις) σταθεροποιούνται στο ~0.63, οι ενδιάμεσες 
τιμές $\alpha = 0.4, 0.6$ (πράσινες αποχρώσεις) στο ~0.59, ενώ οι χαμηλές τιμές 
$\alpha \leq 0.2$ (μπλε αποχρώσεις) στο ~0.54. Η κατακόρυφη διακεκομμένη γραμμή 
στο $k=20$ επισημαίνει την επιλεγμένη βέλτιστη τιμή, η οποία όμως 
θα μπορούσε να ήταν οποιαδήποτε άλλη στο εύρος χωρίς ουσιαστική επίπτωση 
στην απόδοση. Η ανάλυση ευαισθησίας καταδεικνύει ότι το $\alpha$ αποτελεί 
την κρίσιμη υπερπαράμετρο του υβριδικού συστήματος, ενώ το $k$ 
παίζει δευτερεύοντα ρόλο.

\begin{figure}[H]
    \centering
    \includegraphics[width=1\linewidth]{images/chapter5/sensitivity_analysis.png}
    \caption{Ανάλυση ευαισθησίας υπερπαραμέτρων. Αριστερά: η απόδοση αυξάνεται 
μονοτονικά με το $\alpha$, με όλες τις καμπύλες (διαφορετικά $k$) να 
συγκλίνουν. Δεξιά: η απόδοση παραμένει σταθερή για όλα τα $k$, με 
διαστρωμάτωση βάσει του $\alpha$. Το βέλτιστο $\alpha=0.8$ επισημαίνεται με διακεκομμένη γραμμή.}
    \label{fig:experiment2-sensitivity}
\end{figure}

\begin{figure}[H]
    \centering
    \includegraphics[width=1\linewidth]{images/chapter5/test_performance.png}
    \caption{Επικύρωση απόδοσης στο test set: σύγκριση κύριων μετρικών και Success@k για διαφορετικές τιμές k. Το train-test gap είναι ελάχιστο, επιβεβαιώνοντας την ικανότητα γενίκευσης του βέλτιστου μοντέλου.}
    \label{fig:experiment2-test}
\end{figure}

Το Σχήμα \ref{fig:experiment2-test} παρουσιάζει την τελική επικύρωση της απόδοσης στο test set για τη βέλτιστη διαμόρφωση ($\alpha=0.8$, $k=20$). Το αριστερό διάγραμμα δείχνει την εξέλιξη των μετρικών Precision, Recall και F1 καθώς αυξάνεται ο αριθμός των ανακτημένων εγγράφων k. Παρατηρείται η κλασική αντίστροφη σχέση Precision-Recall: η Precision μειώνεται από ~0.75 στο k=1 σε ~0.37 στο k=10, ενώ η Recall αυξάνεται από ~0.18 σε ~0.62. Το μέσο διάγραμμα απεικονίζει τη μετρική Success@k, η οποία αυξάνεται από 0.742 στο k=1 σε 0.907 στο k=10, επιβεβαιώνοντας την υψηλή ικανότητα του συστήματος να εντοπίζει τουλάχιστον ένα σχετικό έγγραφο στα k πρώτα αποτελέσματα. Το δεξί διάγραμμα συγκρίνει την απόδοση μεταξύ train και test set για τις τέσσερις κύριες μετρικές. Η απόκλιση της επίδοσης μεταξύ του σύνολου ελέγχου και του συνόλου εκπαίδευσης είναι ελάχιστη για όλες τις μετρικές (Success@3: 0.867 vs 0.835, Precision@3: 0.629 vs 0.605, Recall@10: 0.598 vs 0.615, Precision@10: 0.361 vs 0.370), υποδηλώνοντας ότι η βέλτιστη διαμόρφωση γενικεύει αποτελεσματικά σε νέα δεδομένα χωρίς σημάδια υπερπροσαρμογής.

\subsection{Συμπεράσματα Πειράματος 2}
\label{subsection:experiment2_conclusions}

Η συστηματική βελτιστοποίηση των υπερπαραμέτρων του αλγορίθμου RRF οδηγεί 
σε σημαντικό εύρημα: η βέλτιστη διαμόρφωση αντιστοιχεί σε υβριδική ανάκτηση με ισχυρή έμφαση στην πυκνή ενσωμάτωση ($\alpha^* = 0.8$, $k^* = 20$), υποδεικνύοντας ότι η συνεισφορά της αραιής ενσωμάτωσης, αν και μικρή (20\%), εξακολουθεί να προσφέρει οριακή βελτίωση έναντι της καθαρά πυκνής ανάκτησης.

Η ανάλυση ευαισθησίας αποκαλύπτει ότι το $\alpha$ αποτελεί την κρίσιμη 
υπερπαράμετρο (μονοτονική αύξηση από $S \approx 0.54$ σε $S \approx 0.63$), 
ενώ το $k$ έχει αμελητέα επίδραση. Οι μετρικές πρώιμης επιτυχίας 
βελτιώνονται δραματικά με αυξανόμενο $\alpha$, ενώ η Recall@10 παραμένει σταθερή (~0.51), υποδηλώνοντας ότι η βελτίωση προέρχεται από καλύτερη κατάταξη των σχετικών εγγράφων στις πρώτες θέσεις. Η επικύρωση στο σύνολο αξιολόγησης επιβεβαιώνει την ικανότητα γενίκευσης με ελάχιστη απόκλιση μεταξύ εκπαίδευσης και ελέγχου.

Η εξαιρετική απόδοση της πυκνής ανάκτησης στο συγκεκριμένο πεδίο μπορεί να ερμηνευτεί μέσω των ιδιαίτερων χαρακτηριστικών του συνόλου δεδομένων. Η στατιστική ανάλυση του SOSum αποκαλύπτει ότι οι ερωτήσεις έχουν διάμεσο μήκος 424 χαρακτήρες (\ref{fig:text_length_distribution}) (περίπου 106 λεκτικές μονάδες), ενώ οι απαντήσεις 300 χαρακτήρες (περίπου 75 λεκτικές μονάδες). Το συνδυασμένο μήκος ενός τυπικού ζεύγους ερώτησης-απάντησης είναι περίπου 180 λεκτικές μονάδες, δηλαδή σημαντικά μικρότερο από το μέγιστο μήκος ακολουθίας των μοντέλων ενσωμάτωσης. Οι κατανομές εμφανίζουν έντονη ασυμμετρία προς χαμηλές τιμές, με την πλειονότητα των εγγράφων να συγκεντρώνεται κάτω από τους 1000 χαρακτήρες. Αυτό επιτρέπει στα μοντέλα ενσωμάτωσης να αποτυπώνουν ολοκληρωμένα το σημασιολογικό περιεχόμενο κάθε εγγράφου στον χώρο αναπαράστασης, χωρίς απώλεια πληροφορίας λόγω συμπίεσης ή ανάγκη για τμηματοποίηση (chunking).

Αντίθετα, σε σύνολα δεδομένων με εκτενή έγγραφα όπως βιβλία, επιστημονικές δημοσιεύσεις, ή τεχνικές αναφορές, το περιεχόμενο υπερβαίνει το μέγιστο μήκος ακολουθίας και πρέπει να διασπαστεί σε τμήματα των 256-512 λεκτικών μονάδων. Η τμηματοποίηση οδηγεί σε απώλεια του ευρύτερου πλαισίου και της συνοχής, καθώς κάθε τμήμα ενσωματώνεται ανεξάρτητα. Σε τέτοιες περιπτώσεις, οι αραιές μέθοδοι διατηρούν το πλεονέκτημα της ακριβούς λεξιλογικής αντιστοίχισης σε ολόκληρο το έγγραφο, και η υβριδική προσέγγιση αναμένεται να επιδείξει μεγαλύτερη σχετική βελτίωση. Αυτή η υπόθεση συνάδει με τα ευρήματα της εργασίας Blended RAG \citep{sawarkar2024blended}, όπου οι υβριδικές μέθοδοι ανάκτησης απέδωσαν σημαντικά καλύτερα σε σύνολα δεδομένων μεγάλου μήκους (όπως το TREC-COVID), επιβεβαιώνοντας την υπεροχή της λεξιλογικής–σημασιολογικής σύζευξης σε αναζήτηση εκτενών εγγράφων.

Επομένως, το πείραμα αποδεικνύει ότι για το πεδίο των τεχνικών ερωτημάτων Stack Overflow, όπου τα έγγραφα είναι σύντομα και σημασιολογικά πυκνά, η υβριδική προσέγγιση με έμφαση στην πυκνή ανάκτηση (80-20) αποδίδει βέλτιστα. Η διαπίστωση ότι η καθαρά πυκνή ανάκτηση δεν υπερτερεί του βέλτιστου υβριδικού μοντέλου υποδηλώνει κάποια συμπληρωματικότητα μεταξύ λεξιλογικής και σημασιολογικής πληροφορίας, αν και σε μικρότερο βαθμό από τον αναμενόμενο. Τα ευρήματα υπογραμμίζουν την ανάγκη αξιολόγησης ανάλογα με το πεδίο εφαρμογής και προσεκτικής ρύθμισης υπερπαραμέτρων, λαμβάνοντας υπόψη τα χαρακτηριστικά του συνόλου δεδομένων (μήκος εγγράφων, σημασιολογική πυκνότητα, ανάγκη για τμηματοποίηση). Η γενίκευση των συμπερασμάτων σε άλλα πεδία με διαφορετικά χαρακτηριστικά απαιτεί περαιτέρω έρευνα. Αξίζει επίσης να σημειωθεί πώς 
