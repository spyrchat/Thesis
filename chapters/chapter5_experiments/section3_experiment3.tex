\section{Πείραμα 3: Αξιολόγηση Παραγωγής Απαντήσεων}
\label{section:experiment3}

Η βελτιστοποίηση του υποσυστήματος ανάκτησης, όπως παρουσιάστηκε στο Πείραμα 2 (\ref{section:experiment2}), δεν διασφαλίζει αναγκαστικά την παραγωγή υψηλής ποιότητας απαντήσεων. Το στάδιο της παραγωγής (generation) συνιστά ανεξάρτητη πηγή σφαλμάτων που μπορεί να επηρεάσει την τελική απόδοση του συστήματος επαυξημένης παραγωγής μέσω ανάκτησης (RAG). Σε αυτό το πείραμα, πραγματοποιείται ολοκληρωμένη αξιολόγηση του συστήματος από άκρη σε άκρη (end-to-end evaluation), εστιάζοντας στην ποιότητα των παραγόμενων απαντήσεων σε τρεις κρίσιμες διαστάσεις: πιστότητα στο ανακτημένο πλαίσιο (faithfulness), συνάφεια με το ερώτημα (relevance), και χρησιμότητα για τον χρήστη (helpfulness).

\subsection{Μεθοδολογία}
\label{subsec:generation_eval_methodology}

\subsubsection{Αρχιτεκτονική Αξιολόγησης}

Για την αξιολόγηση του συστήματος γεννεσιουργίας, υιοθετήθηκε η μεθοδολογία LLM-as-a-Judge \ref{sec:RAGAS}, η οποία αξιοποιεί τις ικανότητες μεγάλων γλωσσικών μοντέλων για την αυτόματη αξιολόγηση της ποιότητας κειμένου. Η προσέγγιση αυτή έχει αποδειχθεί ότι παρουσιάζει υψηλή συσχέτιση με ανθρώπινες κρίσεις και επιτρέπει την κλιμακούμενη αξιολόγηση μεγάλων συνόλων δεδομένων χωρίς την ανάγκη χειροκίνητης προσημείωσης (human annotation). Το σύστημα αξιολόγησης υλοποιήθηκε ως ευφυής πράκτορας με χρήση της βιβλιοθήκης LangGraph, μοντελοποιημένος ως γράφος πεπερασμένων καταστάσεων. Η αρχιτεκτονική του πράκτορα απεικονίζεται στο Σχήμα \ref{fig:eval_agent_architecture} και αποτελείται από τέσσερα διακριτά στάδια επεξεργασίας:

\begin{figure}[H]
    \centering
    \includegraphics[width=0.95\linewidth]{images/chapter5/stategraph1 (5).png}
    \caption{Αρχιτεκτονική ευφυούς πράκτορα για την αξιολόγηση του υποσυστήματος γεννεσιουργίας. Το σύστημα επεξεργάζεται ερωτήματα μέσω αλληλουχίας κόμβων: εκκίνηση, ανάκτηση, γεννεσιουργία, και καταγραφή αποτελεσμάτων.}
    \label{fig:eval_agent_architecture}
\end{figure}

Ο \textbf{κόμβος εκκίνησης} λαμβάνει το ερώτημα του χρήστη και αρχικοποιεί την κατάσταση του γραφήματος. Ο \textbf{κόμβος ανάκτησης (retriever)} χρησιμοποιεί το ερώτημα για να εντοπίσει σχετικά έγγραφα στη διανυσματική βάση δεδομένων, εφαρμόζοντας τη μέθοδο που βρέθηκε στο Πείραμα 2. Ο \textbf{κόμβος γεννεσιουργίας (generator)} λαμβάνει ως είσοδο τόσο το αρχικό ερώτημα όσο και το ανακτημένο πλαίσιο, και παράγει την τελική απάντηση χρησιμοποιώντας το γλωσσικό μοντέλο Llama3.1 8B μέσω της πλατφόρμας Ollama. Τέλος, ο \textbf{κόμβος καταγραφής (benchmark\_logger)} αποθηκεύει τόσο την παραγόμενη απάντηση όσο και τα ανακτημένα έγγραφα για την επακόλουθη αξιολόγηση.

\textbf{Σημαντική παρατήρηση:} Για το συγκεκριμένο πείραμα, το σύστημα λειτούργησε χωρίς μηχανισμό μνήμης (memory-less mode), ώστε να αποφευχθεί η επίδραση προηγούμενων ερωτημάτων στις παραγόμενες απαντήσεις.

\subsubsection{Διαδικασία Αξιολόγησης}

Η αξιολόγηση πραγματοποιήθηκε σε δύο φάσεις. Κατά την πρώτη φάση, το σύστημα επεξεργάστηκε το σύνολο των 500 ερωτημάτων από το dataset SOSum, καταγράφοντας για κάθε ερώτημα την παραγόμενη απάντηση και το σύνολο των ανακτημένων εγγράφων. Κατά τη δεύτερη φάση, χρησιμοποιήθηκε το μοντέλο GPT-5 της OpenAI μέσω προγραμματιστικής διεπαφής (API) ως αξιολογητής (judge), το οποίο βαθμολόγησε κάθε απάντηση σε τρεις διαστάσεις:

\textbf{1. Πιστότητα (Faithfulness):} Αξιολογεί κατά πόσον οι ισχυρισμοί που διατυπώνονται στην απάντηση μπορούν να συναχθούν από το ανακτημένο πλαίσιο, ελαχιστοποιώντας το φαινόμενο των παραισθήσεων (hallucinations).

\textbf{2. Συνάφεια (Relevance):} Εκτιμά το βαθμό στον οποίο η απάντηση αντιμετωπίζει άμεσα το ερώτημα που τέθηκε, χωρίς περιττές πληροφορίες ή παρεκβάσεις.

\textbf{3. Χρησιμότητα (Helpfulness):} Αξιολογεί την πρακτική αξία της απάντησης για τον χρήστη, λαμβάνοντας υπόψη παράγοντες όπως η πληρότητα, η σαφήνεια, και η παρουσία χρήσιμων παραδειγμάτων κώδικα.

Κάθε διάσταση βαθμολογήθηκε στην κλίμακα [1, 5], όπου 0 υποδηλώνει πλήρη αποτυχία και 5 υποδηλώνει ιδανική απόδοση. Έπειτα οι επιμέρους βαθμολογίες διαρούνται με το 5 ώστε να έρθουν στην κλίμακα 0 έως ένα. Η συνολική βαθμολογία υπολογίστηκε ως ο αριθμητικός μέσος όρος των τριών διαστάσεων:

\begin{equation}
\text{Συνολική Βαθμολογία} = \frac{\text{Πιστότητα} + \text{Συνάφεια} + \text{Χρησιμότητα}}{3}
\end{equation}

\subsection{Αποτελέσματα}
\label{subsec:generation_eval_results}

\subsubsection{Περιγραφικά Στατιστικά}

Τα περιγραφικά στατιστικά των μετρικών αξιολόγησης παρουσιάζονται στον Πίνακα \ref{tab:descriptive_stats}. Το σύστημα επιτυγχάνει μέσο όρο συνολικής βαθμολογίας 0.755 ± 0.177, με σημαντικές διαφοροποιήσεις μεταξύ των επιμέρους διαστάσεων.

\begin{table}[H]
\centering
\small
\setlength{\tabcolsep}{10pt}
\renewcommand{\arraystretch}{1.15}
\begin{tabularx}{\textwidth}{@{}lRRRR@{}}
\toprule
\textbf{Μετρική} &
\thead{Πιστότητα} &
\thead{Συνάφεια} &
\thead{Χρησιμότητα} &
\thead{Συνολική\\Βαθμολογία} \\
\midrule
Μέσος όρος   & 0.677 ± 0.209 & 0.873 ± 0.181 & 0.716 ± 0.208 & 0.755 ± 0.177 \\
\bottomrule
\end{tabularx}
\caption{Περιγραφικά στατιστικά των μετρικών αξιολόγησης γεννεσιουργίας (N=500). Οι τιμές παρουσιάζονται ως μέσος όρος ± τυπική απόκλιση. Το μοντέλο GPT-5 χρησιμοποιήθηκε ως αξιολογητής μέσω της μεθοδολογίας LLM-as-a-Judge.}
\label{tab:descriptive_stats}
\end{table}

\subsubsection{Ανάλυση Κατανομών}

Το Σχήμα \ref{fig:score_distributions} απεικονίζει τις κατανομές συχνότητας των βαθμολογιών για κάθε μετρική. Η ανάλυση των κατανομών αποκαλύπτει σημαντικά χαρακτηριστικά της συμπεριφοράς του συστήματος:

\begin{figure}[H]
    \centering
    \includegraphics[width=0.95\linewidth]{images/chapter5/score_distributions.png}
    \caption{Κατανομές συχνότητας των βαθμολογιών αξιολόγησης για τις τέσσερις μετρικές. Οι κόκκινες διακεκομμένες γραμμές υποδεικνύουν τον μέσο όρο ενώ οι πράσινες διακεκομμένες γραμμές τη διάμεσο κάθε κατανομής. Παρατηρείται έντονη ασυμμετρία στη μετρική της Συνάφειας (skewness προς υψηλές τιμές) και ευρύτερη διασπορά στην Πιστότητα.}
    \label{fig:score_distributions}
\end{figure}

\textbf{Πιστότητα (Faithfulness):} Η κατανομή παρουσιάζει μέσο όρο 0.677 με διάμεσο 0.800, υποδεικνύοντας ασυμμετρία προς χαμηλότερες τιμές. Το γεγονός ότι ο μέσος όρος είναι σημαντικά χαμηλότερος της διαμέσου υποδηλώνει την ύπαρξη υποσυνόλου ερωτημάτων όπου το σύστημα παράγει απαντήσεις με σημαντικές παραισθήσεις. Η υψηλή τυπική απόκλιση (0.209) επιβεβαιώνει τη μεταβλητότητα της απόδοσης ανάλογα με τη φύση του ερωτήματος και την ποιότητα του ανακτημένου πλαισίου.

\textbf{Συνάφεια (Relevance):} Η μετρική της Συνάφειας επιτυγχάνει την υψηλότερη βαθμολογία (μέσος όρος 0.873, διάμεσος 1.000), με έντονη συγκέντρωση τιμών στο ανώτερο άκρο της κλίμακας. Αυτό υποδεικνύει ότι το σύστημα είναι αποτελεσματικό στην παραγωγή απαντήσεων που αντιμετωπίζουν το κεντρικό ερώτημα, χωρίς σημαντικές παρεκβάσεις.


\textbf{Χρησιμότητα (Helpfulness):} Η κατανομή της Χρησιμότητας (μέσος όρος 0.716, διάμεσος 0.800) παρουσιάζει παρόμοια χαρακτηριστικά με την Πιστότητα, με σημαντική διασπορά τιμών. Η χαμηλότερη απόδοση σε σχέση με τη Συνάφεια υποδηλώνει ότι ενώ το σύστημα απαντά στα ερωτήματα, η πρακτική αξία των απαντήσεων μπορεί να περιορίζεται από παράγοντες όπως η έλλειψη συγκεκριμένων παραδειγμάτων κώδικα ή η ασαφής διατύπωση.

\textbf{Συνολική Βαθμολογία:} Η κατανομή της συνολικής βαθμολογίας (μέσος όρος 0.755, διάμεσος 0.800) αντικατοπτρίζει την ισορροπία μεταξύ των τριών διαστάσεων, με σημαντική συγκέντρωση τιμών στο εύρος [0.7, 0.9], υποδεικνύοντας γενικά ικανοποιητική αλλά όχι άριστη απόδοση.

Η χαμηλότερη απόδοση στη μετρική της Πιστότητας (0.677) σε σχέση με τις άλλες δύο διαστάσεις αποτελεί κρίσιμο εύρημα. Αυτό υποδηλώνει ότι το κύριο πρόβλημα του συστήματος δεν έγκειται στην κατανόηση του ερωτήματος ή στη γενική δομή των απαντήσεων, αλλά στην τάση του μοντέλου να διατυπώνει ισχυρισμούς που δεν υποστηρίζονται πλήρως από το ανακτημένο πλαίσιο. Αυτή η παρατήρηση καθιστά επιτακτική την ανάγκη για μηχανισμούς αυτοδιόρθωσης και επαλήθευσης των παραγόμενων απαντήσεων, όπως αναλύεται στην επόμενη ενότητα.

\subsection{Πρόταση Βελτίωσης: Μηχανισμός Αυτοδιορθούμενης Παραγωγής}
\label{subsec:self_rag_proposal}

Με βάση την παρατήρηση ότι η χαμηλή πιστότητα αποτελεί το κύριο περιοριστικό παράγοντα της απόδοσης, προτείνεται η ενσωμάτωση μηχανισμού αυτοδιορθούμενης γεννεσιουργίας (Self-Correcting RAG). Η προτεινόμενη αρχιτεκτονική βασίζεται στην ιδέα της επαναληπτικής βελτίωσης μέσω ανατροφοδότησης\cite{asai2023selfrag}.
\subsubsection{Αρχιτεκτονική Self-RAG}

Η προτεινόμενη αρχιτεκτονική απεικονίζεται στο Σχήμα \ref{fig:selfrag_architecture} και εισάγει δύο νέους κόμβους στο γράφημα καταστάσεων που επεκτείνουν τη λειτουργικότητα του βασικού συστήματος.

Ο πρώτος από τους νέους κόμβους που εισάγονται στην αρχιτεκτονική είναι ο κόμβος ανάλυσης ερωτήματος (query\_analyzer). Ο κόμβος αυτός αναλαμβάνει την αποσύνθεση του αρχικού ερωτήματος του χρήστη σε ένα σύνολο ατομικών απαιτήσεων (atomic requirements) που πρέπει να ικανοποιηθούν από την τελική απάντηση. Η διαδικασία αυτή περιλαμβάνει τον προσδιορισμό των βασικών εννοιών (key concepts) που εμπλέκονται στο ερώτημα και τη διατύπωση των βημάτων συλλογιστικής (reasoning steps) που απαιτούνται για την παραγωγή ολοκληρωμένης απάντησης. Η δομημένη ανάλυση που παράγεται από αυτόν τον κόμβο παρέχει στο σύστημα γεννεσιουργίας σαφή καθοδήγηση σχετικά με τα επιμέρους στοιχεία που απαιτείται να συμπεριληφθούν, διευκολύνοντας παράλληλα την επακόλουθη επαλήθευση της πληρότητας και της ορθότητας της παραγόμενης απάντησης.

Ο δεύτερος και κεντρικότερος κόμβος που προστίθεται είναι ο κόμβος αυτοδιορθούμενης γεννεσιουργίας (self\_rag\_generator), ο οποίος υλοποιεί επαναληπτικό βρόχο γεννεσιουργίας-επαλήθευσης-αναθεώρησης. Η λεπτομερής λειτουργία αυτού του μηχανισμού απεικονίζεται στο Σχήμα \ref{fig:selfrag_loop}, όπου παρουσιάζεται η αλληλουχία των σταδίων επεξεργασίας και η ροή ελέγχου του συστήματος.
\begin{figure}[H]
    \centering
    \includegraphics[width=0.95\linewidth]{images/chapter5/selfRAG.png}
    \caption{Προτεινόμενη αρχιτεκτονική ευφυούς πράκτορα με μηχανισμό Self-RAG. Προστίθενται οι κόμβοι query\_analyzer για την αποσύνθεση του ερωτήματος και self\_rag\_generator για την επαναληπτική βελτίωση της απάντησης μέσω επαλήθευσης.}
    \label{fig:selfrag_architecture}
\end{figure}
\begin{figure}[H]
    \centering
    \includegraphics[width=0.6\linewidth]{images/chapter5/selfRAGgenerator.png}
    \caption{Εσωτερικός βρόχος του κόμβου self\_rag\_generator. Το σύστημα παράγει μια αρχική απάντηση, την επαληθεύει έναντι του πλαισίου, και αναθεωρεί επαναληπτικά την απάντηση μέχρι να επιτευχθεί ικανοποιητική πιστότητα ή να εξαντληθεί ο μέγιστος αριθμός επαναλήψεων.}
    \label{fig:selfrag_loop}
\end{figure}

\subsubsection{Λεπτομερής Περιγραφή του Μηχανισμού}

Ο μηχανισμός αυτοδιορθούμενης παραγωγής υλοποιείται μέσω επαναληπτικού βρόχου που συνδυάζει τρία διακριτά στάδια: την παραγωγή απάντησης, την επαλήθευση πιστότητας, και την αναθεώρηση με βάση δομημένη ανατροφοδότηση. Το σύστημα χρησιμοποιεί δύο διαφορετικά πρότυπα οδηγιών (prompt templates) που διαφοροποιούνται ως προς τη λειτουργία τους αλλά διατηρούν κοινή φιλοσοφία. Το πρώτο πρότυπο, που εφαρμόζεται κατά την αρχική γεννεσιουργία, λαμβάνει ως είσοδο το ερώτημα του χρήστη, το ανακτημένο πλαίσιο από τη βάση γνώσης Stack Overflow και την ανάλυση του ερωτήματος που παράγεται από τον κόμβο query\_analyzer. Το δεύτερο πρότυπο, που χρησιμοποιείται κατά τις φάσεις αναθεώρησης, ενσωματώνει επιπρόσθετα την προηγούμενη εκδοχή της απάντησης και δομημένη ανατροφοδότηση που προκύπτει από τη διαδικασία επαλήθευσης. Η ανατροφοδότηση αυτή αποτελείται από τέσσερα στοιχεία: τα συγκεκριμένα ζητήματα που εντοπίστηκαν στην απάντηση, την κατηγοριοποίηση της σοβαρότητάς τους σε τρία επίπεδα (minor, moderate, major), τη συνιστώμενη ενέργεια διόρθωσης, και την αιτιολόγηση της κρίσης του επαληθευτή. Ο επαναληπτικός βρόχος συνεχίζεται μέχρι να ικανοποιηθεί ένα από τα τρία κριτήρια τερματισμού που εφαρμόζονται με συγκεκριμένη προτεραιότητα. Το πρωταρχικό κριτήριο εξετάζει αν η τρέχουσα απάντηση έχει επαληθευτεί ως πιστή στο πλαίσιο, οπότε ο βρόχος τερματίζεται επιτυχώς. Το δευτερεύον κριτήριο λειτουργεί ως μηχανισμός ασφαλείας και τερματίζει τη διαδικασία αν εξαντληθεί ο μέγιστος αριθμός επαναλήψεων, ο οποίος έχει οριστεί σε τρεις προκειμένου να διασφαλιστεί αποδεκτός χρόνος απόκρισης. Το τριτογενές κριτήριο αποτελεί βελτιστοποίηση και επιτρέπει πρόωρο τερματισμό όταν η βαθμολογία εμπιστοσύνης υπερβαίνει το κατώφλι των 0.8 και η σοβαρότητα των εντοπισμένων ζητημάτων χαρακτηρίζεται ως ελάχιστη, αποτρέποντας έτσι υπερβολικές επαναλήψεις για μικροπροβλήματα που δεν επηρεάζουν ουσιαστικά την ποιότητα της απάντησης.
Το σύστημα διατηρεί λεπτομερή μεταδεδομένα για κάθε επανάληψη, καταγράφοντας την παραγόμενη απάντηση, τα αποτελέσματα επαλήθευσης, και την ενέργεια που πραγματοποιήθηκε. Στην τελική έξοδο περιλαμβάνονται δείκτες που υποδεικνύουν αν το σύστημα συνέκλινε σε πιστή απάντηση πριν την εξάντληση των επαναλήψεων και αν εντοπίστηκαν και διορθώθηκαν παραισθήσεις κατά τη διαδικασία.
\subsubsection{Αποτελέσματα Βελτιωμένης Αρχιτεκτονικής}
\begin{table}[H]
\centering
\small
\setlength{\tabcolsep}{10pt}
\renewcommand{\arraystretch}{1.15}
\begin{tabularx}{\textwidth}{@{}lRRRR@{}}
\toprule
\textbf{Μετρική} &
\thead{Πιστότητα} &
\thead{Συνάφεια} &
\thead{Χρησιμότητα} &
\thead{Συνολική\\Βαθμολογία} \\
\midrule
Μέσος όρος   & 0.694 ± 0.206 & 0.857 ± 0.193 & 0.669 ± 0.198 & 0.740 ± 0.180 \\
\bottomrule
\end{tabularx}
\caption{Περιγραφικά στατιστικά των μετρικών αξιολόγησης γεννεσιουργίας χρησιμοποιώντας την αρχιτεκτονική self RAG (N=500). Οι τιμές παρουσιάζονται ως μέσος όρος ± τυπική απόκλιση. Το μοντέλο GPT-5 χρησιμοποιήθηκε ως αξιολογητής μέσω της μεθοδολογίας LLM-as-a-Judge.}
\label{tab:descriptive_stats}
\end{table}
\newpage
\begin{figure}[H]
    \centering
    \includegraphics[width=1\linewidth]{images/chapter5/self_rag_score_distributions.png}
    \caption{Κατανομές συχνότητας των βαθμολογιών αξιολόγησης για τις τέσσερις μετρικές χρησιμοποιώντας την αρχιτεκτονική self RAG.}
    \label{fig:selfrag_score_dist}
\end{figure}

Τα αποτελέσματα αποκαλύπτουν θεμελιώδη συμβιβασμό μεταξύ πιστότητας και πληρότητας. Ο μηχανισμός επαλήθευσης υιοθετεί συντηρητική στρατηγική, απομακρύνοντας προτάσεις που δεν υποστηρίζονται πλήρως από το πλαίσιο, ακόμα και όταν αυτές συμβάλλουν στην πρακτική χρησιμότητα. Αυτό συνάδει με τη βιβλιογραφία για την αφαιρετική περίληψη \cite{maynez2020faithfulness}, όπου συστήματα βελτιστοποιημένα για πιστότητα παράγουν πιο συντηρητικές και λιγότερο πλούσιες περιλήψεις. Η υπόθεση για αυτή τη συμπεριφορά είναι τριπλή: ο επαληθευτής εμφανίζει υπέρμετρη ευαισθησία, η αναθεώρηση προτιμά απομάκρυνση αντί τροποποίησης, και κρισιμότερο, η περιορισμένη ικανότητα του υποκείμενου μοντέλου αποτελεί σημαντικό παράγοντα.

Το σύστημα χρησιμοποιεί το Llama3.1 8B μέσω Ollama, το οποίο εμφανίζει περιορισμούς σε πολύπλοκες εργασίες μετα-γνωστικής φύσης όπως η αυτοεπαλήθευση. Η ασυμμετρία με τον αξιολογητή GPT-5 αποκαλύπτει σημαντική διαφορά δυνατοτήτων συλλογιστικής. Οι ρυθμίσεις (παράθυρο 8192, παραγωγή 2048 λεκτικές μονάδες) επιτρέπουν την επεξεργασία της εισόδου αναθεώρησης (1250-1750 λεκτικές μονάδες), αλλά η γνωστική πολυπλοκότητα σε συνδυασμό με το μικρό μέγεθος μοντέλου οδηγεί σε υποβέλτιστη αξιοποίηση. Μεγαλύτερα μοντέλα επιδεικνύουν ανώτερη απόδοση σε εργασίες αυτοδιόρθωσης \cite{madaan2023selfrefine}, καθώς διαθέτουν ισχυρότερες ικανότητες αιτιολόγησης και καλύτερη διατήρηση πληρότητας κατά την αναθεώρηση. Ο παρατηρούμενος συμβιβασμός μπορεί επομένως να οφείλεται μερικώς στους περιορισμούς του μοντέλου και όχι σε εγγενή αδυναμία της προσέγγισης Self-RAG.

Παρά τη βελτίωση στην πιστότητα (+2.5\%), το σύστημα δεν επιτυγχάνει συνολική βελτίωση λόγω της υποβάθμισης της χρησιμότητας (-6.6\%). Βραχυπρόθεσμα, προτείνεται διαβαθμισμένη επαλήθευση που διακρίνει κρίσιμες από ήπιες αποκλίσεις, στρατηγική τροποποίησης αντί απομάκρυνσης, και ρύθμιση κατωφλίων επαλήθευσης. Μεσοπρόθεσμα, η αναβάθμιση σε ισχυρότερο μοντέλο αναμένεται να βελτιώσει την ισορροπία πιστότητας-χρησιμότητας. Μακροπρόθεσμα, εναλλακτικές αρχιτεκτονικές όπως βαθμολογίες απόδοσης αντί επαλήθευσης, ή υβριδικές προσεγγίσεις με ειδικευμένα μοντέλα επαλήθευσης και γενεσιουργίας, μπορούν να προσφέρουν ευέλικτο έλεγχο του συμβιβασμού μεταξύ πιστότητας και πληρότητας.