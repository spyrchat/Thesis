\section{Αρχιτεκτονικές Επιλογές}
\label{section:design_choices}

Η αρχιτεκτονική του προτεινόμενου συστήματος βασίζεται σε δύο θεμελιώδεις 
αγωγούς που συνιστούν τον πυρήνα της λειτουργικότητάς του. Ο πρώτος 
αγωγός αφορά την εισαγωγή και επεξεργασία δεδομένων (data ingestion 
pipeline), ενώ ο δεύτερος περιλαμβάνει την ανάκτηση πληροφορίας και τη διαδικασία παραγωγής απαντήσεων (retrieval and generation pipeline). Αυτοί οι δύο 
αγωγοί συνδέονται μέσω μιας κοινής υποδομής διαχείρισης διανυσματικών 
αναπαραστάσεων, η οποία υλοποιείται μέσω της διανυσματικής βάσης δεδομένων, 
διασφαλίζοντας τη συνοχή, την αποδοτικότητα και την κλιμακωσιμότητα 
του συστήματος.

Ο αγωγός εισαγωγής δεδομένων ακολουθεί μια σταδιακή προσέγγιση μετασχηματισμού των ακατέργαστων εγγράφων σε δομημένες διανυσματικές ενσωματώσεις (embeddings). Κάθε στάδιο του αγωγού υλοποιείται μέσω εξειδικευμένων δομοστοιχείων που επικοινωνούν μέσω τυποποιημένων διεπαφών, επιτρέποντας την απρόσκοπτη αντικατάσταση ή επέκταση της λειτουργικότητας 
χωρίς επιπτώσεις στον πυρήνα του συστήματος. Αντίστοιχα, ο αγωγός ανάκτησης και παραγωγής ενορχηστρώνεται από έναν ευφυή πράκτορα (intelligent agent) βασισμένο στο πλαίσιο LangGraph. Ο πράκτορας ερμηνεύει τα ερωτήματα των χρηστών, αποφασίζει δυναμικά για την ανάγκη ανάκτησης εξωτερικής πληροφορίας, και συνθέτει συνεκτικές απαντήσεις αξιοποιώντας τις δυνατότητες των μεγάλων γλωσσικών μοντέλων σε συνδυασμό με το ανακτημένο πλαίσιο.

Το σύστημα έχει σχεδιαστεί ακολουθώντας τη φιλοσοφία της διαμόρφωσης ως κώδικα (Configuration as Code), επιτυγχάνοντας τον πλήρη διαχωρισμό μεταξύ διαμόρφωσης και υλοποίησης. Μέσω δηλωτικών αρχείων YAML και μιας εκτελέσιμης τερματικής διεπαφής (CLI), οι χρήστες μπορούν να πειραματιστούν με διαφορετικούς συνδυασμούς παραμέτρων, στρατηγικών τμηματοποίησης, μοντέλων ενσωμάτωσης και τεχνικών ανάκτησης, διασφαλίζοντας την 
επαναληψιμότητα των αποτελεσμάτων και την προσαρμοστικότητα του συστήματος σε διαφορετικά πεδία εφαρμογής.

\begin{figure}[H]
\centering
\includegraphics[width=\linewidth]{images/chapter4/System High Level.png}
\caption{Αρχιτεκτονική του προτεινόμενου συστήματος RAG}
\label{fig:high-level-overview}
\end{figure}

\subsection{Αγωγός Εισαγωγής Δεδομένων}
\label{subsection:data_insertion_pipeline}

Ο αγωγός εισαγωγής δεδομένων αποτελεί την κρίσιμη φάση κατά την οποία τα ακατέργαστα έγγραφα μετασχηματίζονται σε δομημένες διανυσματικές ενσωματώσεις, έτοιμες για αποδοτική ανάκτηση. Το σύστημα υλοποιεί μια επταστασταδιακή διαδικασία που διασφαλίζει την ποιότητα, την ιχνηλασιμότητα και την αναπαραγωγισιμότητα της επεξεργασίας. Η αρχιτεκτονική σχεδιάστηκε με γνώμονα την αρθρωτότητα και την επεκτασιμότητα, επιτρέποντας την απρόσκοπτη ενσωμάτωση νέων πηγών δεδομένων και στρατηγικών επεξεργασίας 
χωρίς την ανάγκη τροποποίησης του κεντρικού πυρήνα του συστήματος.

\textbf{Φόρτωση Παραμέτρων Διαμόρφωσης:} το πρώτο στάδιο αφορά τη φόρτωση 
και επεξεργασία των παραμέτρων διαμόρφωσης. Κάθε σύνολο δεδομένων συνοδεύεται από ένα δηλωτικό αρχείο διαμόρφωσης που καθορίζει τον τύπο του προσαρμοστή (adapter type), τη στρατηγική τμηματοποίησης (chunking strategy), την προέλευση των ενσωματώσεων (embedding provider), τις διαστάσεις των διανυσμάτων, καθώς και πλήθος άλλων παραμέτρων που 
επηρεάζουν τη συμπεριφορά της επεξεργασίας.

\textbf{Ανάγνωση και Επικύρωση Δεδομένων:} το δεύτερο στάδιο περιλαμβάνει την ανάγνωση των ακατέργαστων δεδομένων και την αρχική τους επικύρωση. Το σύστημα διαθέτει μια συλλογή εξειδικευμένων προσαρμοστών (adapters), καθένας από τους οποίους υλοποιεί το αφηρημένο πρότυπο \texttt{DatasetAdapter} και είναι υπεύθυνος για τον χειρισμό συγκεκριμένου τύπου δεδομένων. Οι προσαρμοστές ενθυλακώνουν τη λογική που απαιτείται για την ανάγνωση 
ετερογενών μορφών αρχείων, το φιλτράρισμα και την τυποποίηση των δεδομένων σε μια κοινή αναπαράσταση. Κάθε εγγραφή των ακατέργαστων δεδομένων μετασχηματίζεται αρχικά σε αντικείμενο τύπου \texttt{BaseRow}, το οποίο αποτελεί μια τυποποιημένη αναπαράσταση που περιέχει το κείμενο του εγγράφου, μεταδεδομένα προέλευσης και μοναδικούς αναγνωριστικούς κωδικούς. Στη συνέχεια, τα αντικείμενα \texttt{BaseRow} μετατρέπονται σε έγγραφα LangChain μέσω του προσαρμοστή, εμπλουτίζοντας τα με επιπλέον πληροφορίες πλαισίου 
και δομημένα μεταδεδομένα που διευκολύνουν την μεταγενέστερη ανάκτηση και 
ερμηνεία των αποτελεσμάτων.

Έχοντας ολοκληρώσει τη μετατροπή, τα έγγραφα υποβάλλονται σε αυστηρή διαδικασία επικύρωσης που υλοποιείται από την κλάση \texttt{DocumentValidator}. Η επικύρωση περιλαμβάνει τρεις κατηγορίες ελέγχων: επιβεβαίωση ότι το μήκος κάθε εγγράφου βρίσκεται εντός προκαθορισμένων ορίων, απομάκρυνση περιεχομένου HTML που ενδέχεται να επηρεάσει την ποιότητα των ενσωματώσεων, και ανίχνευση διπλότυπων εγγράφων βασιζόμενη σε κρυπτογραφικές συναρτήσεις κατακερματισμού τύπου SHA-256.

\textbf{Τμηματοποίηση Εγγράφων:} το τρίτο στάδιο αφορά την τμηματοποίηση (chunking) των επικυρωμένων εγγράφων σε μικρότερες σημασιολογικά συνεκτικές μονάδες κειμένου. Η τμηματοποίηση αποτελεί κρίσιμο σημείο στην αρχιτεκτονική, καθώς η επιλογή της κατάλληλης στρατηγικής επηρεάζει άμεσα την ποιότητα της ανάκτησης και την αποδοτικότητα του συστήματος. Η κλάση \texttt{ChunkingStrategyFactory} υποστηρίζει πολλαπλές στρατηγικές 
τμηματοποίησης, καθεμία βελτιστοποιημένη για συγκεκριμένους τύπους περιεχομένου. Η αναδρομική στρατηγική (recursive chunking) εφαρμόζει ιεραρχικό διαχωρισμό βασισμένο σε χαρακτήρες, δοκιμάζοντας διαδοχικά διαφορετικούς διαχωριστές μέχρι να επιτευχθεί το επιθυμητό μέγεθος τμήματος. Η σημασιολογική στρατηγική (semantic chunking) λαμβάνει υπόψη τα όρια των προτάσεων και των παραγράφων, διασφαλίζοντας ότι τα τμήματα διατηρούν 
σημασιολογική ακεραιότητα. Κάθε τμήμα που παράγεται εμπλουτίζεται με εκτεταμένα μεταδεδομένα που περιλαμβάνουν την αναφορά στο πηγαίο έγγραφο, τη θέση εντός του αρχικού κειμένου, το μέγεθος του τμήματος, και ένα ντετερμινιστικό αναγνωριστικό.

\textbf{Παραγωγή Διανυσματικών Ενσωματώσεων:} το τέταρτο στάδιο περιλαμβάνει την παραγωγή διανυσματικών ενσωματώσεων για κάθε τμήμα κειμένου. Το υποσύστημα \texttt{EmbeddingPipeline} υλοποιεί μια ευέλικτη αρχιτεκτονική που υποστηρίζει τρεις στρατηγικές ενσωμάτωσης. Η πυκνή στρατηγική (dense embedding) παράγει διανύσματα υψηλής διάστασης που αιχμαλωτίζουν τη σημασιολογική ομοιότητα μεταξύ κειμένων. Η αραιή στρατηγική (sparse embedding) δημιουργεί αραιά διανύσματα βασισμένα σε στατιστικές λέξεων-κλειδιών με προσεγγίσεις όπως BM25 και SPLADE++. Η υβριδική στρατηγική (hybrid embedding) συνδυάζει και τις δύο προσεγγίσεις, επιτρέποντας στο σύστημα να αξιοποιεί τόσο τη σημασιολογική όσο και τη λεξιλογική ομοιότητα. Το υποσύστημα 
εφαρμόζει ομαδοποίηση των τμημάτων σε δέσμες (batching) για την αποδοτική επεξεργασία και διαθέτει ενσωματωμένη λογική προσωρινής αποθήκευσης (caching) για την αποφυγή επαναϋπολογισμού ενσωματώσεων.\\
\textbf{Αποθήκευση, Επαλήθευση και Καταγραφή:} το πέμπτο στάδιο αφορά την αποθήκευση των επεξεργασμένων τμημάτων μαζί με τις ενσωματώσεις τους στη διανυσματική βάση δεδομένων. Το δομοστοιχείο \texttt{VectorStoreUploader} διαχειρίζεται τη διαδικασία μεταφόρτωσης, διασφαλίζοντας την ταυτοδυναμία (idempotency) μέσω ντετερμινιστικών αναγνωριστικών. Κάθε τμήμα αποθηκεύεται ως ένα σημείο (point) στη βάση δεδομένων που περιέχει το πρωτότυπο κείμενο στο πεδίο δεδομένων (payload), τους διανυσματικούς του εκπροσώπους, και όλα τα σχετικά μεταδεδομένα. Το έκτο στάδιο περιλαμβάνει αυτοματοποιημένους ελέγχους επαλήθευσης (smoke tests) μέσω της κλάσης \texttt{SmokeTestRunner}. Τέλος, το έβδομο στάδιο αφορά την καταγραφή της πλήρους προέλευσης (lineage) 
και του ιστορικού της επεξεργασίας σε μορφή JSON που περιλαμβάνει το πλήρες αρχείο διαμόρφωσης, το αναγνωριστικό έκδοσης κώδικα, εκδόσεις βιβλιοθηκών, χρονοσήμανση σε μορφή ISO 8601, και στατιστικά επεξεργασίας.

\begin{figure}[H]
\centering
\includegraphics[width=0.65\linewidth]{images/chapter4/ingestion-yml.png}
\caption{Τυπικό αρχείο YAML για ορισμό του αγωγού εισαγωγής δεδομένων}
\label{fig:ingestion-yml}
\end{figure}
\subsection{Αγωγός Ανάκτησης Πληροφορίας}
\label{subsection:retrieval_pipeline}

Ο αγωγός ανάκτησης πληροφορίας αποτελεί την κεντρική ενότητα της αρχιτεκτονικής που μεσολαβεί μεταξύ του ερωτήματος του χρήστη και της διανυσματικής βάσης δεδομένων, επιτελώντας την κρίσιμη λειτουργία του ταχέος και ακριβούς εντοπισμού των πιο σχετικών τμημάτων κειμένου. Το σύστημα υλοποιεί μια ευέλικτη αρχιτεκτονική αγωγού επεξεργασίας (retrieval pipeline) που επιτρέπει τη σύνθεση πολλαπλών σταδίων ανάκτησης και βελτίωσης αποτελεσμάτων, διαμορφώσιμη μέσω δηλωτικών αρχείων YAML.

\textbf{Αρχιτεκτονική και Στάδια:} η βασική δομή του αγωγού ακολουθεί μια πεντασταδιακή διαδικασία: επεξεργασία ερωτήματος και παραγωγή διανυσματικών αναπαραστάσεων (query encoding), αρχική ανάκτηση από τη διανυσματική βάση δεδομένων (retrieval), φιλτράρισμα βασισμένο σε κατώφλια βαθμολογίας (filtering), αναδιάταξη για βελτίωση της σειράς των αποτελεσμάτων (reranking), και συναρμολόγηση τελικών εγγράφων (result assembly). Η κεντρική κλάση \texttt{RetrievalPipeline} ενθυλακώνει τη λογική ενορχήστρωσης όλων των 
σταδίων, υλοποιώντας το σχεδιαστικό πρότυπο της αλυσίδας ευθύνης (Chain of Responsibility pattern), όπου κάθε δομοστοιχείο υλοποιεί την αφηρημένη διεπαφή \texttt{RetrievalComponent}.

\textbf{Στρατηγικές Ανάκτησης:} το σύστημα υποστηρίζει τρεις θεμελιώδεις στρατηγικές ανάκτησης. Η πυκνή στρατηγική (dense retrieval) βασίζεται στον υπολογισμό της ομοιότητας συνημιτόνου μεταξύ πυκνών διανυσματικών ενσωματώσεων υψηλής διάστασης, χρησιμοποιώντας αποδοτικούς αλγορίθμους όπως ο αλγόριθμος ιεραρχικού πλοηγήσιμου μικρού κόσμου (HNSW) \cite{malkov2018efficient}. Η αραιή στρατηγική (sparse retrieval) εφαρμόζει προσέγγιση βασισμένη σε στατιστικές λέξεων-κλειδιών όπως το BM25 \cite{robertson2009probabilistic} 
και εξελιγμένα νευρωνικά μοντέλα όπως το SPLADE++ \cite{formal2022splade}. Η υβριδική στρατηγική (hybrid retrieval) συνδυάζει τις δύο προηγούμενες προσεγγίσεις, εκτελώντας παράλληλα πυκνή και αραιή αναζήτηση και συνδυάζοντας τα αποτελέσματα μέσω του αλγορίθμου Reciprocal Rank Fusion.

\textbf{Φιλτράρισμα και Αναδιάταξη:} μετά την αρχική ανάκτηση, η κλάση 
\texttt{ScoreFilter} εφαρμόζει φιλτράρισμα βασισμένο σε κατώφλια βαθμολογίας 
για την απομάκρυνση αποτελεσμάτων χαμηλής ποιότητας. Το στάδιο αναδιάταξης 
(reranking) χρησιμοποιεί εξειδικευμένα μοντέλα νευρωνικών δικτύων μέσω της 
κλάσης \texttt{RerankerComponent}. Σε αντίθεση με τους αρχικούς μηχανισμούς 
ανάκτησης που υπολογίζουν ανεξάρτητες αναπαραστάσεις (αρχιτεκτονική διπλού 
κωδικοποιητή, bi-encoder), τα μοντέλα αναδιάταξης επεξεργάζονται το ερώτημα 
και το έγγραφο από κοινού (αρχιτεκτονική διασταυρούμενου κωδικοποιητή, 
cross-encoder), επιτρέποντας τη μοντελοποίηση λεπτών αλληλεπιδράσεων μέσω 
μηχανισμών προσοχής (attention mechanisms).
\newpage

\begin{figure}[H]
\centering
\includegraphics[width=0.65\linewidth]{images/chapter4/hybrid_retrieval.drawio.png}
\caption{Υβριδικό σύστημα ανάκτησης από διανυσματική βάση δεδομένων}
\label{fig:hybrid-retrieval}
\end{figure}

\begin{figure}[H]
\centering
\includegraphics[width=0.75\linewidth]{images/chapter4/yml-retrieval.png}
\caption{Τυπικό αρχείο YAML για ορισμό του αγωγού ανάκτησης}
\label{fig:retrieval-yml}
\end{figure}

\subsection{Σύστημα Ευφυούς Πράκτορα}
\label{subsection:agent_system}

Το σύστημα ευφυούς πράκτορα αποτελεί το ανώτερο επίπεδο της αρχιτεκτονικής 
που ενορχηστρώνει τη συνολική αλληλεπίδραση με τον χρήστη, λαμβάνοντας 
αποφάσεις σχετικά με την ανάγκη ανάκτησης πληροφορίας και συνθέτοντας 
συνεκτικές απαντήσεις. Το σύστημα υλοποιείται χρησιμοποιώντας το πλαίσιο 
LangGraph, που επιτρέπει τη μοντελοποίηση 
σύνθετων ροών εργασίας ως κατευθυνόμενων γραφημάτων με διαχείριση κατάστασης 
(stateful directed graphs).

Η αρχιτεκτονική βασίζεται στο σχεδιαστικό πρότυπο της μηχανής πεπερασμένων 
καταστάσεων (Finite State Machine), όπου κάθε κατάσταση αναπαριστά την πρόοδο 
επεξεργασίας ενός ερωτήματος. Η κατάσταση του συστήματος ενθυλακώνει το 
αρχικό ερώτημα του χρήστη, τις αποφάσεις δρομολόγησης, το ανακτημένο πλαίσιο, 
την παραγόμενη απάντηση, και το ιστορικό συνομιλίας. Το γράφημα ροής 
εργασίας αποτελείται από τέσσερα κύρια στάδια: ανάλυση και ταξινόμηση 
ερωτήματος όπου το γλωσσικό μοντέλο καθορίζει εάν 
απαιτείται ανάκτηση, ανάκτηση πληροφορίας που ενεργοποιείται υποθετικά, 
παραγωγή απάντησης με προσαρμοστική συμπεριφορά (adaptive prompting) ανάλογα 
με την παρουσία πλαισίου, και διαχείριση ιστορικού συνομιλίας με παράθυρο 
μεταβλητού μεγέθους (sliding window) για πολυστροφικούς διαλόγους.

\section{Τεχνολογικές Επιλογές}
\label{section:technological_choices}

\subsection{Διανυσματική Βάση Δεδομένων}
\label{subsection:vector_database}

Το σύστημα χρησιμοποιεί τη βάση δεδομένων Qdrant, μια λύση 
ανοικτού κώδικα εξειδικευμένη στην αποθήκευση και αναζήτηση διανυσματικών 
αναπαραστάσεων υψηλής διάστασης. Η Qdrant επιλέχθηκε έναντι εναλλακτικών 
όπως η Pinecone, η Weaviate και η Milvus για τέσσερις κύριους λόγους: 
εξαιρετική απόδοση σε εφαρμογές πραγματικού χρόνου με χρόνο αναζήτησης της 
τάξεως των χιλιοστών του δευτερολέπτου, υποστήριξη προηγμένων τεχνικών 
κβαντοποίησης διανυσμάτων που μειώνουν το memory footprint, πλήρη υποστήριξη 
για υβριδική αναζήτηση που συνδυάζει πυκνά και αραιά διανύσματα εντός της 
ίδιας συλλογής, και ευελιξία τοπικής εκτέλεσης χωρίς εξάρτηση από εξωτερικές 
υπηρεσίες cloud. Για την ανάπτυξη, η Qdrant εκτελείται σε Docker container 
με persistent volumes που διασφαλίζουν τη διατήρηση 
των δεδομένων.

\subsection{Πλαίσιο Ανάπτυξης}
\label{subsection:development_framework}

Το LangGraph επιλέχθηκε ως το κεντρικό πλαίσιο 
για την υλοποίηση του πράκτορα, παρέχοντας δηλωτική προσέγγιση (declarative 
approach) για τον ορισμό γραφημάτων ροής εργασίας. Η Python επιλέχθηκε 
ως γλώσσα υλοποίησης λόγω του πλούσιου οικοσυστήματός της σε βιβλιοθήκες 
μηχανικής μάθησης και επεξεργασίας φυσικής γλώσσας, ενώ η Pydantic v2 χρησιμοποιείται για την επικύρωση δεδομένων και τη διασφάλιση συνέπειας μέσω type hints και runtime validation.

\subsection{Υποδομή Γλωσσικού Μοντέλου}
\label{subsection:llm_infrastructure}
Για τη διαδικασία παραγωγής κειμένου, το σύστημα χρησιμοποιεί την πλατφόρμα Ollama σε συνδυασμό με το γλωσσικό μοντέλο Llama3.1 8B. Το Ollama αποτελεί εργαλείο ανοικτού κώδικα που επιτρέπει την τοπική εκτέλεση μεγάλων γλωσσικών μοντέλων με βελτιστοποιημένο κβαντισμό παραμέτρων, διασφαλίζοντας την αυτονομία του συστήματος και την προστασία των δεδομένων. Το Llama3.1 διαθέτει 8 δισεκατομμύρια παραμέτρους και παράθυρο πλαισίου 128,000 tokens και υλοποιεί αυτοπαλίνδρομη παραγωγή κειμένου βασιζόμενο στα ανακτημένα έγγραφα από τη βάση δεδομένων, παράγοντας συνεκτικές απαντήσεις με ελαχιστοποίηση του φαινομένου των παραισθήσεων.