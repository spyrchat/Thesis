\section{Διάρθρωση της Αναφοράς}
\label{section:layout}
Η παρούσα διπλωματική εργασία διαρθρώνεται σε έξι κεφάλαια, τα οποία ακολουθούν συστηματική οργάνωση από τη θεωρητική θεμελίωση έως την πειραματική επαλήθευση των προτεινόμενων μεθόδων:

\begin{itemize}
    \item \textbf{Κεφάλαιο 1 - Εισαγωγή:} Το πρώτο κεφάλαιο εισάγει τον αναγνώστη στην ερευνητική περιοχή των συστημάτων επαυξημένης παραγωγής με ανάκτηση, παρουσιάζοντας το επιστημονικό κίνητρο που υπαγορεύει την ανάγκη ανάπτυξης βελτιωμένων αρχιτεκτονικών. Αναλύεται το πρόβλημα της αναξιοπιστίας των σύγχρονων γλωσσικών μοντέλων και διατυπώνονται οι ερευνητικοί στόχοι που θέτει η εργασία για την αντιμετώπιση των εγγενών περιορισμών αυτών των συστημάτων.
    
    \item \textbf{Κεφάλαιο 2 - Θεωρητικό Υπόβαθρο:} Το δεύτερο κεφάλαιο συγκροτεί το θεωρητικό και τεχνολογικό υπόβαθρο που απαιτείται για την κατανόηση της προτεινόμενης προσέγγισης. Εξετάζονται οι θεμελιώδεις αρχές λειτουργίας των αρχιτεκτονικών μετασχηματιστών, η εξέλιξη των μεγάλων γλωσσικών μοντέλων, οι μέθοδοι διανυσματικής αναπαράστασης σημασιολογικού περιεχομένου, και οι αλγόριθμοι ανάκτησης πληροφοριών που συνθέτουν τα σύγχρονα συστήματα RAG.
    
    \item \textbf{Κεφάλαιο 3 - Κριτική Ανασκόπηση Μεθοδολογιών
Επαυξημένης Παραγωγής μέσω Ανάκτησης:} Το τρίτο κεφάλαιο παρουσιάζει κριτική ανασκόπηση της υπάρχουσας βιβλιογραφίας, αναλύοντας τις μεθοδολογικές προσεγγίσεις που έχουν προταθεί για την επίλυση του προβλήματος της επαυξημένης παραγωγής μέσω ανάκτησης. Εξετάζονται συστηματικά οι αρχιτεκτονικές που έχουν αναπτυχθεί από ερευνητικά κέντρα και οργανισμούς, αξιολογούνται τα πλεονεκτήματα, οι περιορισμοί τους και εντοπίζονται τα κενά που η παρούσα εργασία επιδιώκει να καλύψει.
    
    \item \textbf{Κεφάλαιο 4 - Υλοποίηση:} Το τέταρτο κεφάλαιο αναπτύσσει τη μεθοδολογία που ακολουθήθηκε για τον σχεδιασμό και την υλοποίηση του συστήματος. Περιγράφονται αναλυτικά οι αρχιτεκτονικές επιλογές, οι αλγόριθμοι που εφαρμόστηκαν και η διαδικασία ενορχήστρωσης των επιμέρους δομοστοιχείων σε ένα ενιαίο λειτουργικό σύνολο.
    
    \item \textbf{Κεφάλαιο 5 - Πειράματα και Αποτελέσματα:} Το πέμπτο κεφάλαιο παρουσιάζει το πειραματικό πρωτόκολλο και τα αποτελέσματα της αξιολόγησης. Αναλύονται συστηματικά τα δεδομένα που προέκυψαν από τη σύγκριση διαφορετικών στρατηγικών ανάκτησης, εξετάζεται βελτιστοποίηση του αλγορίθμου RRF και παρουσιάζονται ποιοτικές μετρικές παραγωγής απαντήσεων.
    
    \item \textbf{Κεφάλαιο 6 - Συμπεράσματα και Μελλοντική Εργασία:} Το έκτο και τελευταίο κεφάλαιο συνθέτει τα συμπεράσματα που απορρέουν από την ερευνητική διαδικασία, αξιολογεί την επίτευξη των αρχικών στόχων, αναγνωρίζει τους περιορισμούς της προτεινόμενης προσέγγισης, και υποδεικνύει κατευθύνσεις για μελλοντική έρευνα που θα μπορούσε να επεκτείνει και να βελτιώσει τα αποτελέσματα της παρούσας εργασίας.
    
\end{itemize}