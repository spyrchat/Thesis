\section{Περιγραφή του Προβλήματος}
\label{section:problem_description}

Τα σύγχρονα μεγάλα γλωσσικά μοντέλα έχουν επιδείξει εξαιρετικές ικανότητες στην κατανόηση και παραγωγή φυσικής γλώσσας, καθιστώντας τα πολύτιμα εργαλεία για ένα ευρύ φάσμα εφαρμογών. Ωστόσο, η αξιοπιστία των πληροφοριών που παράγουν παραμένει ένα κρίσιμο και άλυτο πρόβλημα που περιορίζει σημαντικά την εφαρμογή τους σε κρίσιμα συστήματα λήψης αποφάσεων \cite{ji2023survey}.

Το φαινόμενο της παραισθητικότητας (hallucination) αποτελεί την πλέον σημαντική πρόκληση στη χρήση των γλωσσικών μοντέλων. Τα μοντέλα αυτά παράγουν συχνά απαντήσεις που φαίνονται συντακτικά και σημασιολογικά ορθές, αλλά περιέχουν ανακριβείς ή εντελώς κατασκευασμένες πληροφορίες. Σύμφωνα με μελέτες της OpenAI, τα ποσοστά παραισθήσεων στα πρώιμα μοντέλα όπως το GPT-3 έφταναν στο 42\%, ενώ για μικρότερα μοντέλα τα πράγματα γίνονται ακόμα χειρότερα\cite{lin2022truthfulqa}. Έκτοτε έχουν γίνει αξιοσημείωτες προσπάθειες για την ελαχιστοποίηση των παραισθήσεων, καθώς η επιστημονική κοινότητα της τεχνητής νοημοσύνης έχει επικεντρώσει σε αυτόν τον στόχο τις προσπάθειές της. Παρ’ όλα αυτά, ακόμα και σήμερα τα φαινόμενα παραισθήσεων δεν έχουν εξαλειφθεί, με αποτέλεσμα η χρήση σε εξειδικευμένους τομείς όπως η νομική ή η ιατρική να κρίνεται ακατάλληλη, τουλάχιστον χωρίς περαιτέρω διερεύνηση και επικύρωση των παραγόμενων πληροφοριών.

Επιπλέον, τα γλωσσικά μοντέλα λειτουργούν με στατική γνώση που προέρχεται αποκλειστικά από τα δεδομένα εκπαίδευσής τους. Αυτό σημαίνει ότι δεν μπορούν να έχουν πρόσβαση σε πληροφορίες που δημοσιεύθηκαν μετά την ημερομηνία διακοπής της εκπαίδευσής τους (knowledge cutoff), ούτε σε εξειδικευμένες ή ιδιωτικές πηγές δεδομένων που δεν συμπεριλήφθηκαν στο σώμα εκπαίδευσης. Η επανεκπαίδευση ενός μεγάλου μοντέλου όπως το GPT-4 απαιτεί χιλιάδες GPU-hours και εκτιμάται ότι κοστίζει δεκάδες έως εκατοντάδες εκατομμύρια δολάρια \cite{sharir2020cost}, καθιστώντας την πρακτικά ανέφικτη για τακτική ενημέρωση.

Παράλληλα, οι παραδοσιακές μέθοδοι ανάκτησης πληροφοριών που βασίζονται σε λεξικολογική αντιστοίχιση (όπως ο αλγόριθμος BM25) αδυνατούν να κατανοήσουν τη σημασιολογική σχέση μεταξύ διαφορετικών όρων \cite{robertson2009probabilistic}. Αντίστοιχα, τα συστήματα που βασίζονται αποκλειστικά σε σημασιολογική αναζήτηση μέσω πυκνών διανυσματικών αναπαραστάσεων μπορεί να χάσουν σημαντικές λεπτομέρειες όταν η ακριβής αντιστοίχιση όρων είναι κρίσιμη \cite{karpukhin2020dense}. Πρόσφατες μελέτες έχουν δείξει ότι η υβριδική προσέγγιση που συνδυάζει και τις δύο μεθόδους επιτυγχάνει καλύτερα αποτελέσματα από κάθε μέθοδο μεμονωμένα \cite{ma2023fine}.

Η πολυπλοκότητα αυξάνεται περαιτέρω λόγω της ετερογένειας των πηγών δεδομένων που πρέπει να διαχειριστεί ένα σύγχρονο σύστημα πληροφοριών. Επιστημονικές δημοσιεύσεις, τεχνική τεκμηρίωση, φόρουμ συζητήσεων και βάσεις δεδομένων έχουν διαφορετική δομή, ύφος και απαιτήσεις επεξεργασίας \cite{gao2023retrieval}. Ένα ενιαίο σύστημα που μπορεί να διαχειριστεί αποτελεσματικά όλες αυτές τις πηγές απαιτεί σύνθετους μηχανισμούς προσαρμογής και επεξεργασίας.

Τέλος, η έλλειψη ολοκληρωμένων πλαισίων που συνδυάζουν αποτελεσματικά την ανάκτηση πληροφοριών με την ικανότητα παραγωγής κειμένου των γλωσσικών μοντέλων δημιουργεί ένα σημαντικό κενό στην πρακτική εφαρμογή αυτών των τεχνολογιών. Οι περισσότερες υπάρχουσες υλοποιήσεις παραμένουν είτε πειραματικές και δύσκολες στην ανάπτυξη είτε εμπορικές και κλειστού κώδικα, γεγονός που δυσχεραίνει τη δημιουργία προσαρμοσμένων λύσεων.

