\section{Συνεισφορά της Εργασίας}
\label{section:contribution}

Η παρούσα διπλωματική εργασία συνεισφέρει στην επίλυση των προαναφερθέντων προκλήσεων μέσω της ανάπτυξης ενός επεκτάσιμου πλαισίου πειραματισμού για συστήματα επαυξημένης παραγωγής μέσω ανάκτησης, συνοδευόμενου από συστηματική πειραματική αξιολόγηση στο πεδίο των τεχνικών ερωτημάτων μηχανικής λογισμικού.

Το σύστημα που αναπτύχθηκε αποτελεί ένα προσαρμόσιμο και επεκτάσιμο περιβάλλον πειραματισμού που επιτρέπει τη δημιουργία και αξιολόγηση διαφορετικών αγωγών επεξεργασίας (pipelines) μέσω δηλωτικών αρχείων YAML και τερματικής διεπαφής (CLI). Η αρχιτεκτονική βασίζεται σε αφηρημένες διεπαφές που διευκολύνουν την προσαρμογή σε διαφορετικά σύνολα δεδομένων και πεδία εφαρμογής, ενώ παρέχει έτοιμη υποστήριξη για πολλαπλές μεθόδους ανάκτησης (BM25, SPLADE, Dense, Hybrid) από διαφορετικούς παρόχους (OpenAI, VoyageAI, HuggingFace, Google). Η προσαρμοστικότητα του συστήματος επιτρέπει στον ερευνητή να πειραματιστεί με διαφορετικές σχεδιαστικές επιλογές χωρίς σημαντικές τροποποιήσεις στον πυρήνα του κώδικα, διασφαλίζοντας παράλληλα την αναπαραγωγισιμότητα των πειραμάτων.

Πέρα από το πλαίσιο υλοποίησης, η εργασία παρέχει τεκμηριωμένη μεθοδολογία πολυεπίπεδης και ιεραρχικά δομημένης βελτιστοποίησης, από τη μέθοδο ανάκτησης έως την αξιολόγηση από άκρη σε άκρη (end-to-end). Μέσω συστηματικής συγκριτικής αξιολόγησης πέντε μεθόδων ανάκτησης, εξαντλητικής αναζήτησης βέλτιστων υπερπαραμέτρων και αξιολόγησης μηχανισμού αυτοδιορθούμενης παραγωγής απαντήσεων σε τρεις διαστάσεις ποιότητας, η εργασία διερευνά τις σχέσεις μεταξύ των χαρακτηριστικών του συνόλου δεδομένων και της αποτελεσματικότητας των μεθόδων ανάκτησης και παραγωγής. Σε αντίθεση με προσεγγίσεις που επιδιώκουν την ολική βελτιστοποίηση συστημάτων επαυξημένης παραγωγής μέσω ανάκτησης, η παρούσα εργασία εστιάζει στη στοχευμένη βελτιστοποίηση επιμέρους στοιχείων όπου κρίνεται απαραίτητη, τεκμηριώνοντας κάθε σχεδιαστική επιλογή με βάση τις πρακτικές και τους συμβιβασμούς που ορίζει η βιβλιογραφία. Ο συνδυασμός επεκτάσιμου πλαισίου πειραματισμού και συστηματικής πειραματικής μεθοδολογίας στοχεύει να αποτελέσει χρήσιμο οδηγό για την ένταξη τέτοιων συστημάτων σε παραγωγικά περιβάλλοντα, παρέχοντας τόσο την τεχνική υποδομή όσο και την εμπειρική τεκμηρίωση που απαιτείται για τη λήψη τεκμηριωμένων σχεδιαστικών αποφάσεων.