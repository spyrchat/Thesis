\chapter{Εισαγωγή}
\label{section:intro}

Η ανάπτυξη της αρχιτεκτονικής του μετασχηματιστή (Transformer) το 2017 \cite{vaswani2017attention} σηματοδότησε την απαρχή μιας νέας εποχής στην επεξεργασία φυσικής γλώσσας, θέτοντας τα θεμέλια για την εμφάνιση των Μεγάλων Γλωσσικών Μοντέλων. Από τα πρώτα μοντέλα της οικογένειας GPT με εκατομμύρια παραμέτρους \cite{radford2018improving}, η εξέλιξη οδήγησε σε αρχιτεκτονικές δισεκατομμυρίων παραμέτρων όπως το GPT-3 \cite{brown2020language} και το GPT-4 \cite{openai2023gpt4}, τα οποία επέδειξαν πρωτόγνωρες ικανότητες στη σύνθεση κειμένου, την επίλυση προβλημάτων και τη γλωσσική κατανόηση.

Η μετάβαση από τα εξειδικευμένα μοντέλα επεξεργασίας γλώσσας σε γενικής χρήσης συστήματα τεχνητής νοημοσύνης έχει επαναπροσδιορίσει τις δυνατότητες αυτοματοποίησης γνωστικών εργασιών. Τα σύγχρονα LLMs έχουν ενσωματωθεί σε εφαρμογές που εκτείνονται από την αυτοματοποιημένη συγγραφή κώδικα και την επιστημονική έρευνα, μέχρι την εκπαιδευτική υποστήριξη και την ιατρική διάγνωση \cite{singhal2023large}. Η ικανότητά τους να προσαρμόζονται σε νέες εργασίες με ελάχιστα παραδείγματα (few-shot learning) και να εκτελούν σύνθετους συλλογισμούς έχει ανοίξει νέους ορίζοντες στην αλληλεπίδραση ανθρώπου-μηχανής.

Παρά τις εντυπωσιακές τους δυνατότητες, η μετάβαση από πειραματικές εφαρμογές σε παραγωγικά συστήματα αποκαλύπτει θεμελιώδεις περιορισμούς που απειλούν την αξιοπιστία και την ασφάλεια των εφαρμογών. Η τεχνική της Επαυξημένης Παραγωγής μέσω Ανάκτησης έχει προταθεί ως λύση για την άμβλυνση αυτών των περιορισμών, επιτρέποντας στα μοντέλα να αξιοποιούν δυναμικά εξωτερικές πηγές γνώσης κατά την παραγωγή απαντήσεων \cite{lewis2020retrieval}. Συστήματα όπως το RETRO της DeepMind \cite{borgeaud2022improving} και το Atlas της Meta \cite{izacard2022atlas} έχουν αποδείξει ότι η ενσωμάτωση μηχανισμών ανάκτησης μπορεί να βελτιώσει δραματικά την ακρίβεια και την αξιοπιστία των παραγόμενων απαντήσεων.

Η εξέλιξη των συστημάτων επαυξημένης παραγωγής μέσω ανάκτησης (RAG) έχει ακολουθήσει μια πορεία συνεχούς βελτίωσης, από απλούς μηχανισμούς ανάκτησης βασισμένους σε λέξεις-κλειδιά μέχρι σύνθετες αρχιτεκτονικές που συνδυάζουν πολλαπλές στρατηγικές αναζήτησης και προηγμένους αλγορίθμους κατάταξης. Παράλληλα η πρόσφατη ανάπτυξη πλαισίων όπως το LangChain έχει απλοποιήσει την υλοποίηση τέτοιων συστημάτων.

Ωστόσο, η κατασκευή αποτελεσματικών συστημάτων RAG απαιτεί την επίλυση πολύπλοκων τεχνικών προκλήσεων που εκτείνονται πέρα από την απλή σύνδεση ενός μοντέλου με μια βάση δεδομένων. Υπάρχει μια αλυσίδα αλληλοεξαρτούμενων μηχανισμών, που χρειάζονται βελτιστοποίηση προκειμένου να διασφαλιστεί η ποιότητα της ανάκτησης και η αξιοπιστία των παραγόμενων απαντήσεων. 

Η παρούσα διπλωματική εργασία στοχεύει στη συστηματική διερεύνηση και επίλυση αυτών των προκλήσεων μέσω του σχεδιασμού και της υλοποίησης ενός ολοκληρωμένου συστήματος RAG. Η έρευνα εστιάζει στην ανάπτυξη λύσεων, βελτιστοποιώντας κάθε στάδιο της επεξεργασίας από την εισαγωγή των ακατέργαστων δεδομένων μέχρι την παραγωγή της τελικής απάντησης. Για να αποκτήσει η έρευνα πρακτική αξία, η βελτιστοποίηση εφαρμόζεται σε ένα σύνολο δεδομένων τεχνικών ερωτήσεων από το StackOverflow.


\section{Περιγραφή του Προβλήματος}
\label{section:problem_description}

Τα σύγχρονα μεγάλα γλωσσικά μοντέλα έχουν επιδείξει εξαιρετικές ικανότητες στην κατανόηση και παραγωγή φυσικής γλώσσας, καθιστώντας τα πολύτιμα εργαλεία για ένα ευρύ φάσμα εφαρμογών. Ωστόσο, η αξιοπιστία των πληροφοριών που παράγουν παραμένει ένα κρίσιμο και άλυτο πρόβλημα που περιορίζει σημαντικά την εφαρμογή τους σε κρίσιμα συστήματα λήψης αποφάσεων \cite{ji2023survey}.

Το φαινόμενο της παραισθητικότητας (hallucination) αποτελεί την πλέον σημαντική πρόκληση στη χρήση των γλωσσικών μοντέλων. Τα μοντέλα αυτά παράγουν συχνά απαντήσεις που φαίνονται συντακτικά και σημασιολογικά ορθές, αλλά περιέχουν ανακριβείς ή εντελώς κατασκευασμένες πληροφορίες. Σύμφωνα με μελέτες της OpenAI, τα ποσοστά παραισθήσεων στα πρώιμα μοντέλα όπως το GPT-3 έφταναν στο 42\%, ενώ για μικρότερα μοντέλα τα πράγματα γίνονται ακόμα χειρότερα\cite{lin2022truthfulqa}. Έκτοτε έχουν γίνει αξιοσημείωτες προσπάθειες για την ελαχιστοποίηση των παραισθήσεων, καθώς η επιστημονική κοινότητα της τεχνητής νοημοσύνης έχει επικεντρώσει σε αυτόν τον στόχο τις προσπάθειές της. Παρ’ όλα αυτά, ακόμα και σήμερα τα φαινόμενα παραισθήσεων δεν έχουν εξαλειφθεί, με αποτέλεσμα η χρήση σε εξειδικευμένους τομείς όπως η νομική ή η ιατρική να κρίνεται ακατάλληλη, τουλάχιστον χωρίς περαιτέρω διερεύνηση και επικύρωση των παραγόμενων πληροφοριών.

Επιπλέον, τα γλωσσικά μοντέλα λειτουργούν με στατική γνώση που προέρχεται αποκλειστικά από τα δεδομένα εκπαίδευσής τους. Αυτό σημαίνει ότι δεν μπορούν να έχουν πρόσβαση σε πληροφορίες που δημοσιεύθηκαν μετά την ημερομηνία διακοπής της εκπαίδευσής τους (knowledge cutoff), ούτε σε εξειδικευμένες ή ιδιωτικές πηγές δεδομένων που δεν συμπεριλήφθηκαν στο σώμα εκπαίδευσης. Η επανεκπαίδευση ενός μεγάλου μοντέλου όπως το GPT-4 απαιτεί χιλιάδες GPU-hours και εκτιμάται ότι κοστίζει δεκάδες έως εκατοντάδες εκατομμύρια δολάρια \cite{sharir2020cost}, καθιστώντας την πρακτικά ανέφικτη για τακτική ενημέρωση.

Παράλληλα, οι παραδοσιακές μέθοδοι ανάκτησης πληροφοριών που βασίζονται σε λεξικολογική αντιστοίχιση (όπως ο αλγόριθμος BM25) αδυνατούν να κατανοήσουν τη σημασιολογική σχέση μεταξύ διαφορετικών όρων \cite{robertson2009probabilistic}. Αντίστοιχα, τα συστήματα που βασίζονται αποκλειστικά σε σημασιολογική αναζήτηση μέσω πυκνών διανυσματικών αναπαραστάσεων μπορεί να χάσουν σημαντικές λεπτομέρειες όταν η ακριβής αντιστοίχιση όρων είναι κρίσιμη \cite{karpukhin2020dense}. Πρόσφατες μελέτες έχουν δείξει ότι η υβριδική προσέγγιση που συνδυάζει και τις δύο μεθόδους επιτυγχάνει καλύτερα αποτελέσματα από κάθε μέθοδο μεμονωμένα \cite{ma2023fine}.

Η πολυπλοκότητα αυξάνεται περαιτέρω λόγω της ετερογένειας των πηγών δεδομένων που πρέπει να διαχειριστεί ένα σύγχρονο σύστημα πληροφοριών. Επιστημονικές δημοσιεύσεις, τεχνική τεκμηρίωση, φόρουμ συζητήσεων και βάσεις δεδομένων έχουν διαφορετική δομή, ύφος και απαιτήσεις επεξεργασίας \cite{gao2023retrieval}. Ένα ενιαίο σύστημα που μπορεί να διαχειριστεί αποτελεσματικά όλες αυτές τις πηγές απαιτεί σύνθετους μηχανισμούς προσαρμογής και επεξεργασίας.

Τέλος, η έλλειψη ολοκληρωμένων πλαισίων που συνδυάζουν αποτελεσματικά την ανάκτηση πληροφοριών με την ικανότητα παραγωγής κειμένου των γλωσσικών μοντέλων δημιουργεί ένα σημαντικό κενό στην πρακτική εφαρμογή αυτών των τεχνολογιών. Οι περισσότερες υπάρχουσες υλοποιήσεις παραμένουν είτε πειραματικές και δύσκολες στην ανάπτυξη είτε εμπορικές και κλειστού κώδικα, γεγονός που δυσχεραίνει τη δημιουργία προσαρμοσμένων λύσεων.


\section{Συνεισφορά της Εργασίας}
\label{section:contribution}

Η παρούσα διπλωματική εργασία συνεισφέρει στην επίλυση των προαναφερθέντων προκλήσεων μέσω της ανάπτυξης ενός επεκτάσιμου πλαισίου πειραματισμού για συστήματα επαυξημένης παραγωγής μέσω ανάκτησης, συνοδευόμενου από συστηματική πειραματική αξιολόγηση στο πεδίο των τεχνικών ερωτημάτων μηχανικής λογισμικού.

Το σύστημα που αναπτύχθηκε αποτελεί ένα προσαρμόσιμο και επεκτάσιμο περιβάλλον πειραματισμού που επιτρέπει τη δημιουργία και αξιολόγηση διαφορετικών αγωγών επεξεργασίας (pipelines) μέσω δηλωτικών αρχείων YAML και τερματικής διεπαφής (CLI). Η αρχιτεκτονική βασίζεται σε αφηρημένες διεπαφές που διευκολύνουν την προσαρμογή σε διαφορετικά σύνολα δεδομένων και πεδία εφαρμογής, ενώ παρέχει έτοιμη υποστήριξη για πολλαπλές μεθόδους ανάκτησης (BM25, SPLADE, Dense, Hybrid) από διαφορετικούς παρόχους (OpenAI, VoyageAI, HuggingFace, Google). Η προσαρμοστικότητα του συστήματος επιτρέπει στον ερευνητή να πειραματιστεί με διαφορετικές σχεδιαστικές επιλογές χωρίς σημαντικές τροποποιήσεις στον πυρήνα του κώδικα, διασφαλίζοντας παράλληλα την αναπαραγωγισιμότητα των πειραμάτων.

Πέρα από το πλαίσιο υλοποίησης, η εργασία παρέχει τεκμηριωμένη μεθοδολογία πολυεπίπεδης και ιεραρχικά δομημένης βελτιστοποίησης, από τη μέθοδο ανάκτησης έως την αξιολόγηση από άκρη σε άκρη (end-to-end). Μέσω συστηματικής συγκριτικής αξιολόγησης πέντε μεθόδων ανάκτησης, εξαντλητικής αναζήτησης βέλτιστων υπερπαραμέτρων και αξιολόγησης μηχανισμού αυτοδιορθούμενης παραγωγής απαντήσεων σε τρεις διαστάσεις ποιότητας, η εργασία διερευνά τις σχέσεις μεταξύ των χαρακτηριστικών του συνόλου δεδομένων και της αποτελεσματικότητας των μεθόδων ανάκτησης και παραγωγής. Σε αντίθεση με προσεγγίσεις που επιδιώκουν την ολική βελτιστοποίηση συστημάτων επαυξημένης παραγωγής μέσω ανάκτησης, η παρούσα εργασία εστιάζει στη στοχευμένη βελτιστοποίηση επιμέρους στοιχείων όπου κρίνεται απαραίτητη, τεκμηριώνοντας κάθε σχεδιαστική επιλογή με βάση τις πρακτικές και τους συμβιβασμούς που ορίζει η βιβλιογραφία. Ο συνδυασμός επεκτάσιμου πλαισίου πειραματισμού και συστηματικής πειραματικής μεθοδολογίας στοχεύει να αποτελέσει χρήσιμο οδηγό για την ένταξη τέτοιων συστημάτων σε παραγωγικά περιβάλλοντα, παρέχοντας τόσο την τεχνική υποδομή όσο και την εμπειρική τεκμηρίωση που απαιτείται για τη λήψη τεκμηριωμένων σχεδιαστικών αποφάσεων.
\section{Διάρθρωση της Αναφοράς}
\label{section:layout}
Η παρούσα διπλωματική εργασία διαρθρώνεται σε έξι κεφάλαια, τα οποία ακολουθούν συστηματική οργάνωση από τη θεωρητική θεμελίωση έως την πειραματική επαλήθευση των προτεινόμενων μεθόδων:

\begin{itemize}
    \item \textbf{Κεφάλαιο 1 - Εισαγωγή:} Το πρώτο κεφάλαιο εισάγει τον αναγνώστη στην ερευνητική περιοχή των συστημάτων επαυξημένης παραγωγής με ανάκτηση, παρουσιάζοντας το επιστημονικό κίνητρο που υπαγορεύει την ανάγκη ανάπτυξης βελτιωμένων αρχιτεκτονικών. Αναλύεται το πρόβλημα της αναξιοπιστίας των σύγχρονων γλωσσικών μοντέλων και διατυπώνονται οι ερευνητικοί στόχοι που θέτει η εργασία για την αντιμετώπιση των εγγενών περιορισμών αυτών των συστημάτων.
    
    \item \textbf{Κεφάλαιο 2 - Θεωρητικό Υπόβαθρο:} Το δεύτερο κεφάλαιο συγκροτεί το θεωρητικό και τεχνολογικό υπόβαθρο που απαιτείται για την κατανόηση της προτεινόμενης προσέγγισης. Εξετάζονται οι θεμελιώδεις αρχές λειτουργίας των αρχιτεκτονικών μετασχηματιστών, η εξέλιξη των μεγάλων γλωσσικών μοντέλων, οι μέθοδοι διανυσματικής αναπαράστασης σημασιολογικού περιεχομένου, και οι αλγόριθμοι ανάκτησης πληροφοριών που συνθέτουν τα σύγχρονα συστήματα RAG.
    
    \item \textbf{Κεφάλαιο 3 - Κριτική Ανασκόπηση Μεθοδολογιών
Επαυξημένης Παραγωγής μέσω Ανάκτησης:} Το τρίτο κεφάλαιο παρουσιάζει κριτική ανασκόπηση της υπάρχουσας βιβλιογραφίας, αναλύοντας τις μεθοδολογικές προσεγγίσεις που έχουν προταθεί για την επίλυση του προβλήματος της επαυξημένης παραγωγής μέσω ανάκτησης. Εξετάζονται συστηματικά οι αρχιτεκτονικές που έχουν αναπτυχθεί από ερευνητικά κέντρα και οργανισμούς, αξιολογούνται τα πλεονεκτήματα, οι περιορισμοί τους και εντοπίζονται τα κενά που η παρούσα εργασία επιδιώκει να καλύψει.
    
    \item \textbf{Κεφάλαιο 4 - Υλοποίηση:} Το τέταρτο κεφάλαιο αναπτύσσει τη μεθοδολογία που ακολουθήθηκε για τον σχεδιασμό και την υλοποίηση του συστήματος. Περιγράφονται αναλυτικά οι αρχιτεκτονικές επιλογές, οι αλγόριθμοι που εφαρμόστηκαν και η διαδικασία ενορχήστρωσης των επιμέρους δομοστοιχείων σε ένα ενιαίο λειτουργικό σύνολο.
    
    \item \textbf{Κεφάλαιο 5 - Πειράματα και Αποτελέσματα:} Το πέμπτο κεφάλαιο παρουσιάζει το πειραματικό πρωτόκολλο και τα αποτελέσματα της αξιολόγησης. Αναλύονται συστηματικά τα δεδομένα που προέκυψαν από τη σύγκριση διαφορετικών στρατηγικών ανάκτησης, εξετάζεται βελτιστοποίηση του αλγορίθμου RRF και παρουσιάζονται ποιοτικές μετρικές παραγωγής απαντήσεων.
    
    \item \textbf{Κεφάλαιο 6 - Συμπεράσματα και Μελλοντική Εργασία:} Το έκτο και τελευταίο κεφάλαιο συνθέτει τα συμπεράσματα που απορρέουν από την ερευνητική διαδικασία, αξιολογεί την επίτευξη των αρχικών στόχων, αναγνωρίζει τους περιορισμούς της προτεινόμενης προσέγγισης, και υποδεικνύει κατευθύνσεις για μελλοντική έρευνα που θα μπορούσε να επεκτείνει και να βελτιώσει τα αποτελέσματα της παρούσας εργασίας.
    
\end{itemize}
